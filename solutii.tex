INDICATII ȘI RĂSPUNSURI

\section*{Varianta 1}

\section*{Indicații și răspunsuri}
1. Răspuns corect:
a) not $((a<-3)$ or $(a>2))$ or $(a=3)$ or $(a=5)$ or ( $a=9$ ) (Pascal) respectiv
\end{verbatim}

!((a<-3) || (a>2)) || (a\hl{3) || (a}5) || (a==9) (C/C++)

\begin{verbatim}

Indicațiii: Valoarea lui a trebuie să fie mai mare sau egală cu -3 și mai mică sau egală decât 2 sau egală cu 3 sau egală cu 5 sau egală cu 9
2. Răspuns corect: c) $q-1+j$
3. Răspuns corect: $f$ ) info
4. Răspuns corect: $d$ )
\end{verbatim}

Limbajul C++/LimbajulC\\
s=0; i=1;\\
while(i<=n)\\[0pt]
\{s=s+x[p][i];\\
i++;\}

\begin{verbatim}

Limbajul Pascal
\end{verbatim}

s:=0; i:=1;\\
while i<=n do\\
begin\\[0pt]
s:=s+x[p,i];\\
i:=i+1;\\
end;

\begin{verbatim}

\section*{Indicații:}

Fiind vorba de suma elementelor pe linia p, primul indice al elementului din tabloul bidimensional trebuie să fie p. Având n coloane în matrice, cel de-al doilea indice al elementutlui din matrice trebuie să parcurgă toate valorile de la 1 la n.
5.

\section*{Răspuns corect:}
a) 11111
![](https://cdn.mathpix.com/cropped/2025_04_17_46e04c6acd873ea9558dg-251.jpg?height=294&width=625&top_left_y=1672&top_left_x=910)

Indicații: Se observă cu ajutorul desenului de mai sus, că de la nodurile 4 și 5 nu există niciun drum la nodurile 1,2 și 3 . Deci în matrice doar elementele $\mathrm{a}_{41}, \mathrm{a}_{42}, \mathrm{a}_{43}, \mathrm{a}_{51}, \mathrm{a}_{52}$ și $\mathrm{a}_{53}$ vor avea valoarea 0 restul elementelor din matrice având valoarea 1 .
6. Răspuns corect: b) $(3,3,3)$

Indicații: Mulțimea $\mathbf{A}_{\mathbf{2}}$ nu conține elementul 3, deci în produsul cartezian a celor 3 mulțimi nu putem avea ca soluție ( $3,3,3$ ).
7. Răspuns corect: a)
\begin{tabular}{l|l} 
Limbajul C++/LimbajulC \\
a) if (p(x,x)==2) cout<<"prim"; \begin{tabular}{l} 
printf ("prim") ; ;
\end{tabular} & \begin{tabular}{l} 
Limbajul Pascal \\
a) if $p(x, x)=2$ then \\
write ('prim' $) ;$
\end{tabular}
\end{tabular}

Indicații: Funcția calculează numărul de divizori al lui a care sunt mai mici sau egali cu b.
8. Răspuns corect: d) $2,4,0,3,4$
9. Răspuns corect: f) 673656

Indicații: În urma apelului $\mathbf{t}(7,7) \mathrm{x}=7-1=6, \mathrm{y}=6+1=7$ deci se afișează 67 , iar la ieșirea din funcție $\mathbf{y}$ va avea valoarea 6 datorită primului parametru al funcției transmis prin adresă, iar $\mathbf{x}$ va avea valoarea 3 . Se va afișa $\mathbf{x}$ și $\mathbf{y}$ adică 3 și 6 . În urma apelului $\mathbf{t}(6,3) \mathbf{x}=6-1=5$, $y=5+1=6$, se va afișa 56 .
10. Răspuns corect: c)

Limbajul C++/LimbajulC
c) $\operatorname{par}(\mathrm{a}, \mathrm{b})==(\mathrm{b}-\mathrm{a}+1) / 2$

Limbajul Pascal
c) par $(a, b)=(b-a+1)$ DIV 2

Indicații: Pentru verificarea expresiilor, se pot folosi perechi de valori cu aceeași paritate sau de parități diferite.
11. Răspuns corect: $c$ )
\begin{tabular}{l|l} 
Limbajul C++/LimbajulC & \begin{tabular}{l} 
Limbajul Pascal \\
$c) ~$ \\
$c) ~$ \\
$c$
\end{tabular}$\quad=\operatorname{mini}(c, v[i]) ;$
\end{tabular}
12. Răspuns corect: e) amTre
13. Răspuns corect: a)
Limbajul C++/LimbajulC
a) $e[10]=x$;

Limbajul Pascal
a) $e[10]$ := x;
14. Răspuns corect: f) 5120

Indicații: Cu 5 noduri se pot forma $2^{\left(5^{* *}\right) / 2}=2^{10}=1024$ grafuri neorientate distincte. Deci avem 1024 de grafuri distincte în care nodul 2 este adiacent cu nodul 1, alte 1024 de grafuri distincte în care nodul 2 este adiacent cu nodul 3, ..., 1024 de grafuri distincte în care nodul 2 este adiacent cu nodul 6. În total sunt $1024 * 5=5120$ de grafuri distincte.
15. Răspuns corect: c) 5

\section*{Varianta 2}

\section*{Indicații și răspunsuri}
1. Răspuns corect: c) 543212222543445
2. Răspuns corect: d )

Limbajul C++/LimbajulC
d) $(x>y| | x<z) \& \& x>t$

Limbajul Pascal
d) ( $(x>y)$ or $(x<z))$ and ( $x>t$ )
3. Răspuns corect: e)
\begin{tabular}{l|l} 
Limbajul C++/LimbajulC & Limbajul Pascal \\
e) $a[i][j]==a[n+1-j][n+1-i]$ & e) $a[i, j]=a[n+1-j, n+1-i]$
\end{tabular}

Indicații: Se observă că elementele unui tablou bidimensional cu 4 linii și 4 coloane sunt simetrice față de diagonala secundară astfel: $\mathrm{a}_{11}=\mathrm{a}_{44}, \mathrm{a}_{12}=\mathrm{a}_{34}, \mathrm{a}_{13}=\mathrm{a}_{24}, \mathrm{a}_{21}=\mathrm{a}_{43}, \mathrm{a}_{22}=\mathrm{a}_{33}$, $\mathrm{a}_{31}=\mathrm{a}_{42}$, deci putem deduce expresia pe caz general $\mathrm{a}_{\mathrm{i}, \mathrm{j}}=\mathrm{a}_{\mathrm{n}+1-\mathrm{j}, \mathrm{n}+1-\mathrm{i}}$.
4. Răspuns corect: d) 10
5. Răspuns corect: $£$ ) (s.A.y+s.B.y)/2
6. Răspuns corect: b) 0010

0100
1000
0001
Indicații: Se observă că fiecărui tabloul bidimensional îi corespunde o permutare. Permutarea este o funcție f: \{1,2,..,n\}-> \{1,2,..,n\}, bijectivă. Notăm linia 1000 cu 1, linia 0100 cu 2 , linia 0010 cu 3 și linia 0001 cu 4 . Observăm ca permutarea pentru tabloul bidimensional dat este $(3,1,4,2)$ iar permutarea următoare acesteia din punct de vedere lexicografic este $(3,2,1,4)$, corespunzătoare tabloului bidimensional de la punctul b.
7. Răspuns corect: d) 3

Indicații: Funcția calculează numărul de moduri distincte în care poate fi scris un număr $\mathbf{x}$ ca sumă de $\mathbf{y}$ numere naturale strict pozitive. Deci 6 poate fi scris in 3 moduri: 6=1+2+3, $6=1+1+4,6=2+2+2$, același rezultat se obține urmărind apelurile recursive efectuate de funcția $\mathbf{n r}$.
8. Răspuns corect: b) 5, 6, 7, 9, 10
9. Răspuns corect: e) 22

Indicații: În urma apelului $\mathbf{F}(\mathbf{x}, \mathbf{y})$ respectiv $\mathbf{F}(\mathbf{\& x}, \mathbf{y})$, variabila $\mathbf{x}$ este singura variabilă care își modifică valoarea după apel, fiind un parametru transmis prin adresă, deci $\mathbf{y}$ rămâne neschimbat.
10. Răspuns corect: c) 3

Indicații: O soluție: culoarea 1 nodurile 1 , 4 și 6 , culoarea 2 nodurile 3 și 8 , culoarea 3 nodurile 5 și 7 , iar nodul 2 fiind nod izolat propun culoarea 1 (se putea colora și cu 2 și cu $3)$.
11. Răspuns corect: $£) 45$

Indicații: Avem formula: Nrmaxmuchii=(n-p)(n-p+1)/2, unde n reprezintă numărul de noduri iar $\mathbf{p}$ numărul de componente conexe.
12. Răspuns corect: a)

Limbajul C++/LimbajulC
a) cout<<strchr(c, 'd')-c; | printf("\%d", strchr(c, 'd')-c );

Limbajul Pascal
a) write (pos('d', c));
13. Răspuns corect: c) 2021

Indicații: Instrucțiunea for nu produce efecte
14. Răspuns corect: e) combinărilor de 30 de elemente luate câte 5

Indicațiii: Nu contează ordinea în echipă
15. Răspuns corect: f) 14

\section*{Varianta 3}

\section*{Indicații și răspunsuri}
1. Răspuns corect: e) $\mathbf{6 4}$

Indicații: n poate avea valori de la 36 la 99
2. Răspuns corect: a)
3. Răspuns corect: d)

Indicații: Suma tuturor elementelor tabloului a este $\mathbf{0}$. În cadrul programului, se adună toate elementele tabloului a, mai puțin cele pentru care $\mathbf{i}+1=j$. Sunt 19 elemente pentru care $i+j=1$ și fiecare dintre ele are valoarea -1 . Deci $\mathbf{s}+(-19)=0=>\mathbf{s}=19$.
4. Răspuns corect: d)

Indicații: Algoritmul nu este corect implementat. În majoritatea cazurilor, generează ciclare infinită.
5. Răspuns corect: d)

Indicații: Se ține cont de ordinea și modul de transmitere al parametrilor
6. Răspuns corect: b)

Indicații: Funcția dată calculează $\mathbf{x * y}$
7. Răspuns corect: a)

Indicații: Se verifică dacă elementul are loc în stivă
8. Răspuns corect: f)
9. Răspuns corect: e)

Indicații: Numărul poate avea 1, 2 sau 3 cifre
10. Răspuns corect: c)
11. Răspuns corect: e)

Indicații: Numărul valorilor se calculează direct prin formula n* (n-1) /2
12. Răspuns corect: c)
13. Răspuns corect: a)

Indicații: Arcele $(2,1),(2,3)$ și $(2,4)$ au extremitatea inițială nodul 2 (cu gradul exterior 0 ), iar arcele $(3,1)$ și $(4,1)$ au extremitatea finală nodul 1 (cu gradul interior 0)
14. Răspuns corect: b)

Indicații: Numărul ciclurilor hamiltoniene dintr-un graf complet cu n noduri este:
( $\mathrm{n}-1$ ) ! / 2 .
15. Răspuns corect: d)

Indicații: Afirmațile 1,4 și 5 sunt adevărate.

\section*{Varianta 4}

\section*{Indicații și răspunsuri}
1. Răspuns corect: d)

Indicații: Îndeplinesc condiția cerută expresiile 1, 2 și 4
2. Răspuns corect: f)

Indicații: două drepte paralele au aceeași pantă
3. Răspuns corect: e) $a+b+c-f(a, f(c, b))$

Indicații: $a+b+c-\max (a, b, c)=a+b+c-\max (a, \max (b, c))$
4. Răspuns corect: a) -1 2 2-1 -1 1

Indicații: Mecanismul de transmiterea parametrilor.
5. Răspuns corect: e) $\mathbf{1 5 2 0 0 4}$

Indicații: Se mută grupurile de câte $\mathbf{2}$ cifre. Datorită numărului impar de cifre a lui $\mathbf{n}$, în m apare o cifră în plus.
6. Răspuns corect: d) 10239

Indicații: Algoritmul determină baza minimă $\mathbf{x}$ în care îl consideră pe $\mathbf{n}$-ul inițial și îl transformă în baza 10
7. Răspuns corect: e)

Indicații: Linia $\mathbf{n - j + 1}$ din matricea a devine coloana $\mathbf{j}$ în matricea b
8. Răspuns corect: c)
9. Răspuns corect: b)

Indicații: Indicii sunt de la 1 la n; parcurgere liniară a vectorului.
10. Răspuns corect: f) Inserează șirul $\mathbf{t}$ în șirul $\mathbf{s}$, începând cu poziția $\mathbf{k}$
11. Răspuns corect: a) 0

Indicații: Graful este tare conex
12. Răspuns corect: f) ( $1,3,5,2,1,2)$

Indicații: $(3,2,1,5,1,1)$ - are număr impar de noduri de grad impar
$(5,1,6,4,5,3)$ și $(1,1,1,1,1,6)$ - au un nod cu grad prea mare
$(1,1,1,1,2,2)$ - nu poate fi conex deoarece are doar 4 muchii
( $2,1,3,1,0,1$ ) are un nod izolat
13. Răspuns corect: b) 8

Indicații: $\mathbf{3 n + 2 \leq n ( n - 1 ) / 2}$
14. Răspuns corect: c) 3

Indicații: Numerele 12, 16 și 18 generează arbori cu 8 frunze
15. Răspuns corect: d) $\mathbf{2 5}$

Indicații: $C_{5}^{2}+C_{5}^{3}+C_{5}^{4}$

\section*{Varianta 5}

\section*{Indicații și răspunsuri}

Răspuns corect: e)
Răspuns corect: b)
Indicații: Operatorul \% (mod) nu funcționează pe tipul real
Răspuns corect: c) dmtr
Indicații: La ștergerea unei litere, vecina din dreapta îi va lua locul și nu va mai fi eliminată

Răspuns corect: d) Graful G conține cel puțin un ciclu
Indicații: Graful aciclic maximal cu 100 de noduri este un arbore și are 99 de muchii.
Răspuns corect: d) 30
Indicații: Numarul de valori 1 din matricea de adiacență este egal cu numărul de arce
Răspuns corect: e) 5
Indicații: Nodurile terminale au gradul 1, restul nodurilor având gradul 3
Răspuns corect: b)
Răspuns corect: a) ( $1,21,13,23,33,17,27)$
Indicații: Șirul trebuie să fie sortat în funcție de cifra unităților
Răspuns corect: c)
Răspuns corect: b) 12
Răspuns corect: a) 3 și 3
Indicații: Parametrul transmis prin valoare nu se modifică, pe când cel transmis prin referință/adresă, da.

Răspuns corect: a)
Indicații: Atribuirea este corectă între două variabile de același tip RECORD/struct
Răspuns corect: e) 15
Indicații: Matricea de adiacență are valori 0 pe diagonala principală. Dintre celelalte 6 elemente, 4 trebuie să conțină valori 1, ordinea nefiind importantă. Așadar numărul de grafuri este $C_{6}^{4}$.

Răspuns corect: e)
Răspuns corect: b) Graful G este un graf hamiltonian

\section*{Varianta 6}

\section*{Indicații și răspunsuri}
1. Răspuns corect: b) 8

Indicații: numărul este par ( $\mathrm{k}^{*} 2$ ).

Răspuns corect: a) aranjamentelor
Indicații: elementele ce formează o soluție sunt distincte iar ordinea lor în cadrul unei soluții contează.
11.

Răspuns corect: d) 60
Indicații: $C_{5}^{2} * C_{4}^{2}$
12.

Răspuns corect: c) (a+b)\%2==0
Indicații: suma a două numere de aceeași paritate este un număr par .
Răspuns corect: a) aticamatica

Răspuns corect: b) 9
Indicații: se obține un nod izolat
Răspuns corect: d) (1,2,2,1,2,2)
Indicații: suma gradelor tuturor nodurilor unui graf neorientat este un număr par.
Răspuns corect: b)12

Răspuns corect: b) 3
Indicații: graful va conține un circuit elementar cu toate nodurile grafului și încă două arce care au extremitatea inițială în același nod.

Răspuns corect: b) Suma elementelor de pe diagonala secundară a tabloului a
Indicații: pentru ca un element să se găsească pe diagonal secundară tr ebuie ca suma dintre indicele liniei și cel al coloanei să fie $\mathrm{n}+1$.
9. Răspuns corect: b) 4

Indicații: $\mathrm{x}=\mathrm{a}[1][2]+\mathrm{a}[3][4]=1+3=4$
10. Răspuns corect: a) aranjamentelor
dicații: elementele ce formează o soluție sunt distincte iar ordinea lor în cadrul unei
11.

Răspuns corect: d) 60
Indicații: $C_{5}^{2} * C_{4}^{2}$
Răspuns corect: c) determinarea elementului maxim din șir
13. Răspuns corect: a) p.x*p.y>0

Indicații: în aceste cadrane abscisa și ordonata au același semn.
14. Răspuns corect: b) 90

Indicații: toate nodurile îl au ca "tată" pe nodul etichetat cu 10.
15. Răspuns corect: b) 2

\section*{Varianta 7}

\section*{Indicații și răspunsuri}
1. Răspuns corect: b) ! $(x<-4 \| x>-1)\|!(x<1 \| x>4)\|!(x<10) \quad(C++/ C)$ respectiv b) $\operatorname{not}((x<-4)$ or $(x>-1))$ or $\operatorname{not}((x<1)$ or $(x>4))$ or $\operatorname{not}(x<10) \quad$ (Pascal)
2. Răspuns corect: c) floor(5.19) == floor(5.91) ( $\mathrm{C}++/ \mathrm{C}$ ) respectiv c) $\operatorname{trunc}(5.19)=\operatorname{trunc}(5.91) \quad$ (Pascal)
3. Răspuns corect: d) ( $3,4,10,17,46$ )
4.

Răspuns corect:
e) 3 (valorile comparate cu x fiind 12, 18, 17 )
5. Răspuns corect:
f) $3(5,6,7)$
6.

Răspuns corect: d) $20\left(\mathrm{C}_{6}{ }^{3}=20\right)$
7. Răspuns corect:
e) 6
8.

Răspuns corect: f) f.close(); (C++) sau fclose(f); (C) sau close(f); (Pascal)
9.

Răspuns corect: c) 9
10.

Răspuns corect: f) ((c.p1+c.p2)*0.8+c.medbac*0.2)>=5.0
11.

Răspuns corect: d) void cifre (unsigned n, unsigned \&prim, unsigned \&ult) (C++) void cifre (unsigned n, unsigned *prim, unsigned *ult) (C) procedure cifre ( n : longint; var prim,ult: byte); (Pascal)
12. Răspuns corect: f) 10 şi 1
13. Răspuns corect: c) 15 și $210(\mathrm{nr}$ minim arce $=\mathrm{nr}$ vârfuri; $\mathrm{nr} \operatorname{maxim}=\mathrm{n} *(\mathrm{n}-1))$
14. Răspuns corect: f) 1 şi 2
15. Răspuns corect: d) 63

\section*{Varianta 8}

\section*{Indicații și răspunsuri}
1. Răspuns corect: e) !((x>=-4 \&\& $x<=-1)\|(x>=1 \& \& x<=4)\|(x>=10)) \quad(C / C++)$ respectiv e) $\operatorname{not}((x>=-4$ and $x<=-1)$ or ( $x>=1$ and $x<=4$ ) or ( $x>=10)$ ) (Pascal)
2. Răspuns corect: c) floor(5.19) $==$ floor(5.91) $\quad(\mathrm{C} / \mathrm{C}++)$ respectiv c) trunc(5.19)=trunc(5.91) (Pascal)
3. Răspuns corect: f) $(3,4,7,10,12,17,18,20,46)$
4. Răspuns corect: e) $(3,4,10,17,46)$ şi $(7,10,12,18,20)$
5. Răspuns corect: c) 9
6. Răspuns corect: d) 70
7. Răspuns corect: e) 512
8. Răspuns corect: d) f.close(); (C++) sau fclose(f); (C) sau close(f); (Pascal)
9. Răspuns corect: b) '9’
10. Răspuns corect: f) (e.sex=='F' || e.sex=='f') \&\& (e.dn.l==7 \&\& e.dn.z<=10) (C/C++)
respectiv ((e.sex='F') or (e.sex='f')) and (e.dn.l=7) and (e.dn.z<=10) (Pascal)
11. Răspuns corect: c) return suma(n); (C/C++) respectiv c) suma(n); (Pascal)
12. Răspuns corect: f) 56 (numărul minim de muchii se obține pentru două componente conexe, fiecare graf complet cu 8 vârfuri; $m=8 * 7 / 2=28$ reprezintă numărul de muchii pentru fiecare dintre aceste componente)
13. Răspuns corect: e) 9 (graful, având 10 vârfuri, gradul maxim al unui vârf poate fi n1=9)
14. Răspuns corect: c) Graful este (slab) conex
15. Răspuns corect: d) 33

\section*{Varianta 9}

\section*{Indicații și răspunsuri}

\section*{1. Răspuns corect:b)}

Indicații: verificare directă a fiecărei variante folosind pentru no valoare cu cel putin 4 cifre
2. Răspuns corect: f)

Indicații: se ț̦ine cont de precedența operatorilor și de regulile de negare a unei expresii logice
3. Răspuns corect: d)

Indicații: ș̦irul inițial are 8 caractere, pentru că la fiecare iterație se elimină un caracter. Mai departe, se analizează caracterele afișate : primul caracter este 'a', prin urmare al doilea caracter din șirul inițial trebuie să fie 'b' ș.a.m.d
4. Răspuns corect: d)

Indicații:un arbore cu 4 noduri are 3 muchii, este conex și fără cicluri

\section*{5. Răspuns corect:d)}

Indicații:calcul direct al valorilor din matrice, apoi se numără câte dintre elemente sunt valori prime

\section*{6. Răspuns corect: b)}

Indicații: se analizează soluțiile din enunț: soluția care are pe prima poziție valoarea 1 va fi prima generată, apoi urmează soluțiile care au pe prima poziție valoarea 2 ; soluția care conține valoarea 3 pe poziția a 2 -a va fi generată înaintea celei care conține pe poziția a 2 a valoarea 8 ; soluția care are pe prima poziție valoarea 3 va fi generată ultima
7. Răspuns corect: e)

Indicații: (1,2,3,1), (1,3,5,1), (3,4,5,3), (1,2,3,5,1), (1,3,4,5,1), (1,2,3,4,5,1)

\section*{8. Răspuns corect:d)}

Indicații: componentele tare conexe sunt alcătuite din următoarele mulțimi de noduri \{1,2,3,6\},\{4\},\{5\},\{7\}

\section*{9. Răspuns corect:e)}

Indicații:lanțurile de lungime maximă se află între nodurile 9 și 12,9 și 8,9 și 11,10 și 12 , 10 și 8, 10 și 11
10. Răspuns corect: e)

Indicații: se verifică dacă distanța de la centrul cercului la originea sistemului de coordonate este mai mică decât raza cercului
11. Răspuns corect: a)
12. Răspuns corect: a)
13. Răspuns corect: b)

Indicații: variabila globală y este vizibilă în toate funcțiile și va fi modificată de fiecare apel al acestora. Variabila globală x nu va fi modificată, deoarece în fiecare dintre funcț̦ii există un parametru cu același nume, iar apelul $g(x)$ va prelua pe segmentul de stivă doar valoarea lui x .
14. Răspuns corect: d)

Indicații:secvența determină numărul de numere cu cel mult 4 cifre care au numărul divizorilor număr par; se știe că doar pătratele perfecte au număr impar de divizori
15. Răspuns corect: e)

Indicații:funcția verifică (utilizând Divide et Impera) dacă vectorul este sortat strict crescător între pozițiile 2 și 5

\section*{Varianta 10}

\section*{Indicații și răspunsuri}
1. Răspuns corect: c)

Indicații: verificare directă, ținând cont de precedența operatorior
2. Răspuns corect: d)

Indicații:verificare directă a fiecărei variante de răspuns

\section*{3. Răspuns corect: b)}

Indicații: Algoritmul determină ultima cifră a numărului $x^{y}$. Cum ultima cifră a lui $x$ este 7 , ultima cifră a puterilor lui $x$ va fi $7,9,3,1$, apoi se repetă. Pentru y sunt posibile 90 de valori (de la 10 la 99 ), înseamnă că 22 dintre acestea vor determina pentru z valoarea 1.
4. Răspuns corect:d)
5. Răspuns corect: a)

Indicații: $f(f(775125)+f(97917))=f(5+7)=f(12)=-1$
6. Răspuns corect: f)

Indicații: primele 11 soluții sunt: $\{1\},\{1,2\},\{1,2,3\},\{1,2,3,4\},\{1,2,3,4,5\}$, \{1,2,3,4,5,6\}, \{1,2,3,4,5,6,7\}, \{1,2,3,4,5,7\},\{1,2,3,4,6\}, \{1,2,3,4,6,7\}, \{1,2,3,4,7\}
7. Răspuns corect: e)

Indicații: nodurile 1 și 3 au gradul 3 și sunt adiacente, prin urmare pentru a obține un graf eulerian este necesar ca acestea să ajungă să aibă gradul 4.
8. Răspuns corect: b)

Indicații: cele 16 muchii pot determina o componenta conexă cu 17 noduri (arbore), deci rămân 13 noduri izolate și prin urmare numărul minim de componente conexe este 14. Pentru a determina numărul maxim de componente conexe, se caută graful complet cel mai mare care are cel mult 16 muchii. $\mathrm{K}_{6}$ are 15 muchii, deci vom avea o componentă conexă cu 7 noduri, 23 de noduri izolate, adică 24 componente conexe
9. Răspuns corect: c)

Indicații: graful conține arcele: $(2,4),(2,6),(2,8),(2,10),(3,6),(3,9),(4,8),(5,10)$
10. Răspuns corect: b)
11. Răspuns corect: $\mathbf{c}$ )

Indicații: fiecare apel recursiv va gestiona propria variabilă locală i
12. Răspuns corect: d)

Sunt 12 valori afișate: 192021223038464544433527
13. Răspuns corect: f)

Indicații: verificare directă
14. Răspuns corect: d)

Indicații:după executarea subprogramului, vectorul a conține valorile (1,6,15,20,15,6,1)
15. Răspuns corect: e)

Indicații:secvența determină suma cifrelor în baza $b=3$. Pentru a determina suma maximă, îincercăm să considerăm cât mai multe cifre 2 (cifra maximă în baza 3): $2 * 1+2 * 3+2 * 9+2 * 27=80$

\section*{Varianta 11}

\section*{Indicații și răspunsuri}
1. Răspuns corect e)
2. Răspuns corect c )
3. Răspuns corect f)

Indicație: Este suficient să fie calculat radical de ordinul 3 din n pentru a afla câte cuburi mai mici sau egale decât n există.
4. Răspuns corect d)
5. Răspuns corect e)
6. Răspuns corect c)
7. Răspuns corect a)

Indicație: Pentru ca un graf neorientat cu n noduri să fie conex, numărul minim de muchii necesare este n-1.
8. Răspuns corect f)
9. Răspuns corect d)

Indicație: Suma gradelor nu trebuie sa fie egală cu 2n-2 (n=numărul vârfurilor)
10. Răspuns corect e)
11. Răspuns corect b)

Indicație: La sumă se adună doar numerele divizibile cu 3, iar condiția de oprire a recursivității este când i este egal cu 3*n.
12. Răspuns corect b)
13. Răspuns corect a)

Indicație: pentru $\mathrm{b}=29$ si $\mathrm{b}=30$.
Subprogramul calculează numărul valorilor naturale impare din intervalul [a, b].
14. Răspuns corect b)

Indicație: Calculează $C_{n}^{k}$
15. Răspuns corect b)

Indicație: Calculează ultima cifră a lui $x^{y}$

\section*{Varianta 12}

\section*{Indicații și răspunsuri}
1. Răspuns corect b)

Indicație: numărul muchiilor unui graf neorientat complet este $n *(n-1) / 2$
2. Răspuns corect a)

Indicație: condiția ca un element să se situeze pe diagonala secundară a unui tablou bidimensional este $\mathrm{i}+\mathrm{j}=\mathrm{n}+1$ (unde $\mathrm{i}, \mathrm{j}$ reprezintă indicii de linie ș̦i coloană ai elementului
3. Răspuns corect d)
4. Răspuns corect f)
5. Răspuns corect a)
6. Răspuns corect e)
7. Răspuns corect c )
8. Răspuns corect d)
9. Răspuns corect a)

Indicație: Suma gradelor trebuie sa fie egala cu 2n-2 (n=numărul vârfurilor)
10. Răspuns corect d)
11. Răspuns corect b)
12. Răspuns corect e)
13. Răspuns corect c )
14. Răspuns corect d)

Indicație: Valoarea calculată reprezintă numărul divizorilor pozitivi ai variabilei n
15. Răspuns corect c )

\section*{Varianta 13}

\section*{Indicații și răspunsuri}
1. Răspuns corect c) $\mathbf{2}$

Indicații: corecte sunt variantele 3 și 4
2. Răspuns corect e) $2 \mathbf{1 4}$
3. Răspuns corect e) 112
4. Răspuns corect b) 246
5. Răspuns corect a) 4

63
6. Răspuns corect c ) neenUB
7. Răspuns corect d) $\mathbf{1 2 2 5}$

Indicații: 49+48+...+1=50*49/2= 1225 (suma Gauss)

\section*{8. Răspuns corect b) dcafe; dcbaf}
9. Răspuns corect f) 350

Indicațiii: meniu= felul întâi + felul doi sau
meniu $=$ felul întâi + felul doi + desert
10. Răspuns corect c) 31

Indicații: Graful dat are 14 muchii. $\mathrm{K}_{10}$ are 45 muchii. 45-14=31 muchii trebuie adăugate pentru a obține un $\mathrm{K}_{10}$
11. Răspuns corect a) 0
12. Răspuns corect f) $\mathbf{3 0}$

Indicații: Într-un arbore binar se face diferența între fiul stâng și fiul drept. Sunt 5 configurații posibile * 6 etichetări diferite $=30$ cazuri
13. Răspuns corect b) 5

Indicațiii: descendenții nodului 4 sunt: 3, 6, 11, 13, 15
14. Răspuns corect d) $\mathbf{5}^{2} \hat{a}^{\wedge T M} \mathbf{2}^{276}$

Indicații: Fie $\mathbf{n}$ numărul de noduri din graf. Un nod este izolat și cu restul se construiesc grafuri neorientate cu $\mathrm{n}-1$ noduri ( $2^{(\mathrm{n}-1)} \hat{\mathrm{a}}^{\operatorname{TmM}(n-2) / 2}$ cazuri). Orice nod
![](https://cdn.mathpix.com/cropped/2025_04_17_46e04c6acd873ea9558dg-271.jpg?height=55&width=1242&top_left_y=2355&top_left_x=374)
15. Răspuns corect c) $p \hat{a}^{\pi T M} n^{3}$

\section*{Varianta 14}

\section*{Indicații și răspunsuri}
1. Răspuns corect c) 3
2. Răspuns corect d) 31
3. Răspuns corect a) $5 \mathbf{3 0}$ 570
4. Răspuns corect b) 2600
5. Răspuns corect c) 81114
6. Răspuns corect e) EBPU-UPB
7. Răspuns corect d) 4950

Indicații: ( $\mathrm{n}-1$ ) $+(\mathrm{n}-2)+\ldots+1=\mathrm{n} *(\mathrm{n}-1) / 2$ (suma Gauss), unde n este numărul de elemente al tabloului unidimensional
8. Răspuns corect b) 13122

Indicații: $\mathbf{2} \hat{a}^{\wedge \text { rm }} 3^{8}{ }^{8}{ }^{\text {rTM }} 1$
9. Răspuns corect a) $1212 ; 4322$
10. Răspuns corect b) 15
11. Răspuns corect c) 50
12. Răspuns corect d) 50

Indicații: Pentru ca arborele binar să aibă înălțime minimă se ocupă fiecare nivel $x$ cu câte $2^{x}$ noduri, unde $x$ este de la 0 la 5 (rădăcina este pe nivelul 0). Pe nivelul 5 avem 32 frunze. S-au folosit $2^{6}-1=63$ noduri. Pe nivelul 6 se adaugă încă 37 de noduri. Astfel, deduc că rămân 13 frunze pe nivelul 5, deoarece 18 noduri au câte doi fii și un nod are un fiu (32-18-1=13 frunze), iar pe nivelul 6 sunt 37 frunze. Total: 50 frunze.
13. Răspuns corect d) 5

Indicații: $\mathrm{L}_{1}: 10,1,5,7 ; \mathrm{L}_{2}: 10,1,5,12 ; \mathrm{L}_{3}: 10,4,3,11$;
$L_{4}: 10,4,3,13 ; L_{5}: 10,4,3,15$;
14. Răspuns corect d) $5^{2} \hat{a}^{\wedge T M} \mathbf{2}^{552}$

Indicații: Fie $n$ numărul de noduri din graf. Un nod este izolat și cu restul se construiesc grafuri orientate cu n-1 noduri ( $4^{(n-1)} \hat{a}^{\text {arm }(n-2) / 2}$ cazuri $=2^{(n-1)} \hat{a}^{\text {arm }(n-2)}$ cazuri). Orice nod poate fi ales ca nod izolat. Total grafuri orientate nâtm $\mathbf{2}^{(\mathrm{n}-1)} \hat{\mathrm{a}}^{\wedge T \mathrm{~m}(\mathrm{n}-2)}$
15. Răspuns corect a) $O$ ( $n$ )

\section*{Varianta 15}

\section*{Indicații și răspunsuri}
1. Răspuns corect f) 90
2. Răspuns corect b) n și i>0
3. Răspuns corect a)
(Limbajul C/C++)
if(a>b \&\& $a \% 2==0| | b>=a \quad \& \& b \% 2==0) c=a ;$
else c=b;
(Limbajul Pascal)
if ( $(a>b)$ and $(a \bmod 2=0))$ or ( $(b>=a)$ and ( $b \bmod 2=0)$ )
then $\mathrm{c}:=\mathrm{a}$
else c:=b;
4. Răspuns corect e) 8
5. Răspuns corect c) $(5,8,4,0,4,5,3,6,7,8)$
6. Răspuns corect a)
(Limbajul C/C++)
$i<j \& \& i+j<n+1$
(Limbajul Pascal)
(i<j) and (i+j<n+1)
7. Răspuns corect f) 2
8. Răspuns corect e) 3
9. Răspuns corect d) 24
10. Răspuns corect a) 2
11. Răspuns corect c) 2 b 4 d 4
12. Răspuns corect b) diarrafetbdul
13. Răspuns corect f) -8
14. Răspuns corect b) 20
15. Răspuns corect a) 9

\section*{Varianta 16}

\section*{Indicații și răspunsuri}
1. Răspuns corect d) 31
2. Răspuns corect a)
(Limbajul C/C++)
n și i=i-1;
(Limbajul Pascal)
n și i:=i-1;
3. Răspuns corect a)
(Limbajul C/C++)
if (a>b \&\& a\%2==0 \&\& b\%2==0) c=a;
if (a>b \&\& $a \% 2==0 \quad \& \& b \% 2!=0) \quad c=b ;$
(Limbajul Pascal)
if $(a>b)$ and $(a \bmod 2=0)$ and $(b \bmod 2=0)$ then $c:=a ;$
if $(a>b)$ and $(a \bmod 2=0)$ and $(b \bmod 2<>0)$ then $c:=b ;$
4. Răspuns corect f) 44
5. Răspuns corect a) $1,2,6,8,10$
6. Răspuns corect
(Limbajul C/C++) c) $i+j==n+2$
(Limbajul Pascal) c) $i+j=n+2$
7. Răspuns corect a) $\mathbf{x} \in(-\infty, \mathbf{- 1 0}) \cup[\mathbf{1 0}, \mathbf{1 0 0})$
8. Răspuns corect e) 2
9. Răspuns corect b) 23
10. Răspuns corect f) 3
11. Răspuns corect a) 2 c 3 d 4
12. Răspuns corect c) eAiunieeUIa
13. Răspuns corect c) 7892
14. Răspuns corect d) 20
15. Răspuns corect e) 6284

\section*{Varianta 17}

\section*{Indicații și răspunsuri}
1. Răspuns corect b) 2
2. Răspuns corect d) 4
3. Răspuns corect c) 90
4. Răspuns corect a) 1
5. Răspuns corect f) 1110

Indicații: numărul ciclomatic $=m-n+p$
6. Răspuns corect e) oli 2020
7. Răspuns corect d) Bucuresti 2020 ADMIS
8. Răspuns corect a) 864
9. Răspuns corect f) 4
10. Răspuns corect f) nici o valoare
11. Răspuns corect a) $\frac{3}{4} \cdot(\boldsymbol{n}+1) \cdot n$
12. Răspuns corect c) int/int/integer
13. Răspuns corect f) $\mathbf{2 0 1 4}$
14. Răspuns corect c) $O\left(2^{n}\right)$
15. Răspuns corect $a) O(n \cdot \log (n))$

\section*{Varianta 18}

\section*{Indicații și răspunsuri}
1. Răspuns corect b) $\mathbf{2}$
2. Răspuns corect d) 4
3. Răspuns corect a) 0.83
4. Răspuns corect a) 1
5. Răspuns corect f) 6

Indicații: $2^{15}=2^{\mathrm{n}(\mathrm{n}-1) / 2}=32768$
6. Răspuns corect f) 0202 iloP
7. Răspuns corect d) Politehnica Bucuresti 2020XXXXXXXXXXXX
8. Răspuns corect f) 16460640
9. Răspuns corect c$) \mathrm{O}(\mathrm{n})$

Indicațiii: Algoritmul de interclasare
10. Răspuns corect c) $\theta\left(2^{\mathrm{n}}\right)$

Indicații: Turnurile din Hanoi.
11. Răspuns corect a) $\frac{5}{4} \cdot(n+1) \cdot n$
12. Răspuns corect $c$ ) double/double/real
13. Răspuns corect e) 343401
14. Răspuns corect a) $O(n+m)$
15. Răspuns corect d) $O\left(n^{2}\right)$

\section*{Varianta 19}

\section*{Indicații și răspunsuri}
1. Răspuns corect: b) 45
2. Răspuns corect: e) $\mathbf{2}$ instrucțiuni

Indicații: pentru nâ\%ói se execută doar cele două atribuiri.
3. Răspuns corect: f) $\mathbf{4}$ componente

Indicații: $\{1\},\{2,3,4,5\},\{6\},\{7\}$
4. Răspuns corect: d) 6
5. Răspuns corect: $c$ )

Limbajul C++: int $\& x$,int $y$; Limbajul C: int *x, int $y$;
Limbajul Pascal: var x:integer; y:integer;
6. Răspuns corect: b) $9^{138}$

Indicații: Sunt $3^{n(n-1) / 2}$ grafuri complete orientate. Pentru n=24 avem $3^{276}$
7. Răspuns corect: a) 20v
8. Răspuns corect: f) 1

Indicații: Se șterge de exemplu muchia $(1,2)$.
9. Răspuns corect: e) $\mathbf{1 6}$
10. Răspuns corect: $a) O(m \cdot \log (n))$

Indicații: Avem o structură repetitivă cu valori de la 1 la $\mathbf{m}$, în interiorul căreia avem o căutare binară - deci $O(m \cdot \log (n))$
11. Răspuns corect: c$) \mathrm{O}(\mathrm{n})$

Indicați: Citirea vectorului are complexitatea $O$ ( $n$ ), subprogramul are complexitatea $\mathrm{O}(\log (\mathrm{n})$ ). Prin urmare ordinul de complexitate al secvenței este O(n).
12. Răspuns corect: b) $\mathbf{4}^{\mathbf{1 9}} \mathbf{- 1}$

Indicații: Sunt $1+2+2^{2}+2^{3}+\ldots+2^{37}$ noduri, adică $2^{38}-1=4^{19}-1$
13. Răspuns corect: a) $42 \quad 72 \quad 1521518$
14. Răspuns corect: b) 1001997

Indicații: Fie $\mathbf{v}=[\mathrm{n}, \mathrm{n}-1, \ldots, 2,1]$ unde n se află pe poziția $\mathbf{1}, \mathrm{n}-1$ pe poziția 2 , 1 pe poziția $n$. Pentru fiecare element de pe poziția i, iâ\%od2 se execută $3+2$ (i1) pași. Sunt în total $3(n-1)+2[(n(n-1) / 2]$ pași, adică $(n-1)(n+3)$ pași. Pentru $n=6$ avem $5(6+3)=45$ pași, iar pentru $n=1000$ se execută $999(1000+3)=999 \times 1003=1001997$ pași.
15. Răspuns corect: a) 15

Indicațiii: valorile afișate vor fi: 111212112131121 deci în total instrucțiunea de decizie se execută de 15 ori.

\section*{Varianta 20}

\section*{Indicații și răspunsuri}
1. Răspuns corect: b) $\mathbf{2}$ apeluri

Indicații: $\mathbf{f}(\mathbf{7 2 0} \mathbf{2})$ și $\mathbf{f ( 1 2 0 , 1 )}$.
2. Răspuns corect: b) c , a

Indicații: În stivă se rețin valorile variabilelor locale (variabila a) și valorile parametrilor transmiși prin valoare (variabila c).
3. Răspuns corect: c) $O\left(\mathbf{n}^{2}\right)$
4. Răspuns corect: b) Limbajul C/C++: ! ( $x^{*} y+y-3 * x-3>=0$ )

Limbajul Pascal: NOT ( $\mathbf{x} * \mathrm{y}+\mathrm{y}-3 * \mathrm{x}-3>=0$ )
Indicații: $x>-1$ și $y<3$ implică $(x+1)(y-3)<0$ echivalent cu $x y+y-3 x-3<0$ echivalent cu! ( $x^{*} y+y-3 * x-3>=0$ ) în limbajul C++, respectiv NOT ( $x^{*} y+y-3 * x-3>=0$ ) în limbajul Pascal.
5. Răspuns corect: a) 3

Indicații: $\{0,1,2,3,4\},\{0,1,4,5\},\{0,2,3,5\}$
6. Răspuns corect: b) aib

Indicații: Soluțiile care au două vocale în ordinea generării sunt: abe, abi, abu, ace, aci, acu, aeb,aec,aib,aic,aie....
7. Răspuns corect: c) 7

Indicații: $2+2+2+2+2+2 ; 2+2+2+3+3 ; 2+2+3+5 ; 2+3+7 ; 2+5+5 ; 3+3+3+3 ; 5+7$
8. Răspuns corect: a) \{biologie, mate, info\};

Indicații: Solutiile în ordinea generării sunt: \{fizica, biologie, chimie\}; \{fizica, biologie, mate\}; \{fizica, biologie, info\}; \{fizica,chimie, mate\}; \{fizica, chimie, info\}; \{fizica, mate, info\}; \{biologie, chimie, mate\};\{biologie, chimie, info\};\{biologie, mate, info\}; \{chimie, mate, info\}
9. Răspuns corect: c) $\mathbf{2}^{\mathrm{k}}$
10. Răspuns corect: a) linia $\mathbf{5}$, coloana 7

Indicații: Elementul de pe linia i, coloana j are valoarea (i-1) m+j. 123=3m+3 deci $m=40$. Prin urmare sunt 40 elemente pe o linie $167=40 * 4+7$. Deci $i=5, j=7$.
11. Răspuns corect: e) n

Indicații: Cazul cel mai defavorabil este atunci când $\mathbf{n}$ este număr prim. În acest caz, în afara structurii repetitive sunt 2 instrucțiuni de atribuire iar în cadrul structurii repetitive sunt $\mathbf{n - 2}$ instrucțiuni de incrementare. În total $\mathbf{n}$ instrucțiuni.
12. Răspuns corect: e) 246
13. Răspuns corect: b) 443

Indicații: Subprogramul calculează recursiv $x_{1}^{n}+x_{2}^{n}$ unde $\mathrm{s}=\mathrm{x}_{1}+\mathrm{x}_{2}$ iar $\mathrm{p}=\mathrm{x}_{1} * \mathrm{x}_{2}$.
Prin urmare $x_{1}^{4}+x_{2}^{4}=82$ dacă $\mathrm{x}_{1}+\mathrm{x}_{2}=4$ și $\mathrm{x}_{1} * \mathrm{x}_{2}=3$, atunci $\mathrm{x}_{1}=3$ și $\mathrm{x}_{2}=1$ deci $3^{4}+1=82$.
14. Răspuns corect: b) $n=6$; $k=1$

Indicații: Se construiește un tablou bidimensional care are elemente cu valoarea 1 pe primele $\mathbf{k}$ diagonale paralele cu cele două diagonale, în rest, elemente cu valoarea 2.
15. Răspuns corect: c) $\alpha=j ; \quad \beta=4-j-k$

\section*{Varianta 21}

\section*{Indicații și răspunsuri}
1. Răspuns corect d)
2. Răspuns corect d)
3. Răspuns corect c )

Indicație: Atât în Limbajul Pascal cât și în Limbajul C/C++, ; este instrucțiunea vidă.
4. Răspuns corect d)
5. Răspuns corect c )
6. Răspuns corect c)
7. Răspuns corect d)
8. Răspuns corect c)
9. Răspuns corect d)
10. Răspuns corect b)
11. Răspuns corect d)
12. Răspuns corect d)

Indicație: Funția va returna 0 dacă toate elementele vectorului sunt în ordine descrescătoare. În evaluarea expresiei $\mathrm{v}[\mathrm{n}-1]<\mathrm{v}[\mathrm{n}] \| \mathrm{f}(\mathrm{n}-1)$ are loc o scurtcircuitare, dacă o singură dată relația $\mathrm{v}[\mathrm{n}-1]<\mathrm{v}[\mathrm{n}]$ este adevărată funcția va returna valoarea 1 (true) și se întrerupe apelul recursiv.
13. Răspuns corect d)

Indicație: Numărul grafurilor parțiale este egal cu numărul submulțimilor mulțimii muchiilor adică $2^{4}$.
14. Răspuns corect a)

Indicație: formula este $\sum_{i=1}^{10} \sum_{j=1}^{i} \sum_{k=1}^{j} 1=\sum_{i=1}^{10} \sum_{j=1}^{i} j=\sum_{i=1}^{10}(1+2+. .+i)=$ $\sum_{i=1}^{10} \frac{i(i+1)}{2}=\frac{1}{2} \sum_{i=1}^{10}\left(i^{2}+i\right)=220$
15. Răspuns corect d)

\section*{Varianta 22}

\section*{Indicații și răspunsuri}
1. Răspuns corect c )

Indicație: Numărul de frunze este egal cu numărul de factori primi din descompunerea numărului.
2. Răspuns corect b)
3. Răspuns corect c )
4. Răspuns corect d)
5. Răspuns corect b)
6. Răspuns corect a)
7. Răspuns corect d)

\section*{Indicație:}
a) Deoarece nodul 7 are gradul 6, el este adiacent cu toate celelalte noduri, deci nu este posibil ca un nod sa aibă gradul zero.
b) Deoarece nodul 7 are gradul 6, el este adiacent cu toate celelalte noduri. Nodul 6 are gradul 5, aceasta înseamnă că el este adiacent cu 7 și încă 4 noduri, deci trebuie să existe 4 noduri care să aibă minim gradul 2 și nu există.
c) Deoarece nodul 7 are gradul 6, el este adiacent cu toate celelalte noduri. Pentru ca nodul 5 să poată avea gradul 2 trebuie ca încă un nod să aibă minim gradul 2 .
e) Deoarece nodul 7 are gradul 5, el este adiacent cu 5 noduri și nu pot exista două noduri care să aibă gradul 0 .
f) Deoarece nodul 7 are gradul 3, el este adiacent cu 3 noduri. Deoarece nodul 6 are gradul 2, el este adiacent cu 7 si cu încă un nod. Dar nu mai există un nod de grad 2.
8. Răspuns corect d)
9. Răspuns corect b)
10. Răspuns corect d)
11. Răspuns corect a)
12. Răspuns corect d)
13. Răspuns corect b)

\section*{Indicație:}

Pentru n=3 avem un singur ciclu hamiltonian 1231.

Pentru n=4, îl intercalăm pe 4 în toate modurile posibile și obținem ciclurile hamiltoniene: 1423 1, 1243 1, 12341.
Presupunem ca în graful cu n noduri avem $\frac{(n-1)!}{2}$ cicluri hamiltoniene. În graful cu $\mathrm{n}+1$ noduri, intercalând pe $\mathrm{n}+1$ în toate modurile posibile obținem $\mathrm{n} \cdot \frac{(n-1)!}{2}=\frac{n!}{2}$ cicluri hamiltoniene.

\section*{14. Răspuns corect d)}

Indicație: Funcția va returna 0 dacă toate elementele vectorului sunt în ordine descrescătoare. În evaluarea expresiei $\mathrm{v}[\mathrm{n}-1]<\mathrm{v}[\mathrm{n}] \| \mathrm{f}(\mathrm{n}-1)$ are loc o scurtcircuitare, dacă o singură dată relația v[n-1]<v[n] este adevărată funcția va returna valoarea 1 (true) și se întrerupe apelul recursiv.

\section*{15. Răspuns corect d)}

\section*{Varianta 23}

\section*{Indicații și răspunsuri}
1. Răspuns corect: c) $b+a / 10!=b \% c * a / c$ (limbaj $C++/ C)$
b+a div $10<>b$ mod $c * a \operatorname{div} c$ (limbaj Pascal)
2. Răspuns corect: f) 11

Indicații: suma gradelor trebuie să fie un număr par. Pentru $\mathrm{n}=13$ și $\mathrm{d}=11$ suma gradelor ar fi 143.
3. Răspuns corect: c) 7

Indicații: $i=1 ; i=2 ; i=4 ; i=8 ; i=16 ; i=32 ; i=64$
4. Răspuns corect: e) 8

Indicații: se elimină cifrele, dar nu cele care sunt precedate de o cifră ștearsă
5. Răspuns corect: b) $n *(n-1) / 2$

Indicații: se execută $(n-1)+(n-2)+\ldots+2+1$ comparații
6. Răspuns corect: c ) -4

Indicații: $\mathrm{x}=15$; $\mathrm{x}=14 ; \mathrm{x}=7 ; \mathrm{x}=2 ; \mathrm{x}=1 ; \mathrm{x}=-4$;
7. Răspuns corect: d) 13

Indicații: Se generează: 1003, 1012, 1021, 1030, 1102, 1111, 1120, 1201, 1210, 1300, 2002, 2011, 2020
8. Răspuns corect: a) suma elementelor de sub diagonala principală exclusiv elementele diagonalei principale
9. Răspuns corect: e) h.g.c [2]
10. Răspuns corect: e) $1,4,5,6,8$

Indicații: lanțurile elementare de lungime 3 sunt: $(1,2,3,4),(1,2,3,6)(1,2,7,5)(8,2,3,4)$ (8,2,3,6) (8,2,7,5) (3,2,7,5) (6,3,2,7) (4,3,2,7)
Nodurile $1,4,5,6$ și 8 apar în câte 3 lanțuri elementare, celelalte noduri apar de mai multe ori.
11. Răspuns corect: f) 631321

Indicații: Primul "for" atribuie tabloului: 051321
$a[a[6]]=2 * 6 \% 7 \quad a[1]=5$
$a[a[5]]=2 * 5 \% 7 \quad a[2]=3$
$a[a[4]]=2 * 4 \% 7 \quad a[3]=1$
$a[a[3]]=2 * 3 \% 7 a[1]=6$
12. Răspuns corect: c) $\mathbf{s}(\mathbf{2 0 2 0}, \mathbf{2})=\mathbf{4}$ și reprezintă suma exponenților divizorilor primi din descompunerea în factori primi a numărului 2020
13. Răspuns corect: d) 1792

Indicații: Sunt $2^{6}=64$ grafuri cu 4 noduri (nodurile 3, 4, 5, 6).
Dacă există muchia [1,2] atunci 2 se conectează cu 3 sau 4 sau 5 sau 6 . Deci în acest caz avem 4 variante.
Dacă 1 se conectează cu 3 sau 4 sau 5 sau 6 atunci și 2 se conectează cu două dintre nodurile $3,4,5,6$. În acest caz avem $4 * 6=24$ variante.
În total avem 4+24=28 variante de conectare pentru 1 și 2. În total avem 28*64=1792 grafuri cu proprietatea cerută.
14. Răspuns corect:f) 3112210219216
15. Răspuns corect: f) Aplicând metoda de sortare prin inserție se poate obține ca etapă intermediară tabloul $v=(1,3,4,2,5,7,6)$
Indicații: Dacă se parcurge tabloul de la primul element la ultimul, prin inserție avem: $(2,3,4,5,1,7,6)(1,2,3,4,5,7,6)(1,2,3,4,5,6,7)$
Bubble Sort: prima traversare: se schimbă 4 cu $2,5 \mathrm{cu} 1,7$ cu 6 și se obține tabloul ( $3,2,4,1,5$, 6, 7); a doua traversare: se schimbă 3 cu 2,4 cu 1 și se obține tabloul ( $2,3,1,4,5$, 7); a treia traversare: se schimbă 3 cu 1 și se obține tabloul ( $2,1,3,4,5,6,7$ ); a patra traversare: se schimbă $\mathbf{2}$ cu 1 și se obține tabloul sortat
b) interclasare: dacă se interclasează $(2,3,4,5)$ cu $(1,6,7)$, se compară 1 cu 2 și 1 devine primul în tablou, apoi $2 \mathrm{cu} 6,3 \mathrm{cu} 6$ etc.
Dacă se interclasează $(2,3,4)$ cu $(1,5,6,7)$ se compară 1 cu 2 si 1 devine primul în tablou, apoi $2 \mathrm{cu} 5,3 \mathrm{cu} 5$ etc.
c) prin selecția minimului/maximului se fac cel mult $\mathbf{n - 1}$ interschimbări
d) la prima parcurgere se compară 3 cu 2 și minimul devine 2 . Tabloul devine (1, 4, 2,5,3,7,6). La a doua parcurgere minimul inițial este 4, apoi devine $\mathbf{2}$ și se compară cu 3, iar 2 este plasat pe a doua poziție a tabloului.
e) După prima traversare se obține tabloul (3, 2, 4, 1, 5, 6, 7)

\section*{Varianta 24}

\section*{Indicații și răspunsuri}
1. Răspuns corect: d$)$ număr natural impar de o singură cifră

Indicații: $\mathbf{n \%} \mathbf{2 = = 1}$ este adevărată pentru numere naturale impare
2. Răspuns corect: f) $\mathbf{i}+j=n+1$
3. Răspuns corect: d) este hamiltonian dar nu eulerian

Indicații: fiecare nod are gradul 9, care nu este un număr par
4. Răspuns corect: c) a[i]-a[i-1]!=d (C++/C) respectiva[i]-a[i-1]<>d (Pascal)
5. Răspuns corect: b) 2

Indicații: cele mai lungi lanțuri elementare sunt (5, 7, 2, 3, 6), (5, 7, 2, 3, 4) și au în mijloc nodul 2
6. Răspuns corect: a) 1326

Indicații: se compară la fiecare pas ultima cifră a numărului a cu ultima cifră a numărului $\mathbf{b}$ și cea mai mică dintre acestea este adăugată numărului $\mathbf{p}$.
$a=11357$
b= 1426
7. Răspuns corect: c) 19

Indicații: se determină maxim pentru toți indicii, mai puțin pentru ultimul indice
8. Răspuns corect: c) orientat cu 4 noduri și 6 arce

Indicații: Matricea are 4 linii și 4 coloane, deci graful are 4 noduri. Matricea nu este simetrică, deci nu este graf neorientat. Are 6 elemente nenule, deci are 6 arce.
9. Răspuns corect: e) noram și nramo

Indicații: după ordonarea alfabetică a literelor cuvântului roman se obține amnor. Daca nu le ordonează și consideră ca primă soluție cuvântul dat se obține răspunsul a
10. Răspuns corect: c) o rama alba

Indicații: Se elimină spațiile din șir și se verifică dacă este palindrom. Singurul care nu este palindrom este oramaalba
11. Răspuns corect: d) 14

Indicații: $\mathrm{f}(3)=\mathrm{f}(2)+2 * f(0)=3+2 * 1=5$
$f(2)=f(1)+2 * f(-1)=1+2=3$ deci $f(3)$ are 4 apeluri
$f(5)=f(4)+2 * f(2) \quad f(2)=f(1)+2 * f(-1)$
$f(4)=f(3)+2 * f(1) \quad f(3)=f(2)+2 * f(0) \quad f(2)=f(1)+2 * f(-1)$
$f(5)$ și $f(3)$ nu se numără pentru că sunt apeluri din programul principal. Se cere numărul de autoapeluri!
12. Răspuns corect: c) $\mathbf{1 4}$

Indicații: se generează tabloul
\begin{tabular}{rrrr}
1 & 2 & 3 & 4 \\
5 & 6 & 7 & 3 \\
8 & 9 & 6 & 2 \\
10 & 8 & 5 & 1
\end{tabular}
13. Răspuns corect: f) 112 și 166

Indicații: $\mathrm{f}(95)=\mathrm{f}(1+\mathrm{f}(97))=\mathrm{f}(110)=112$;
$\mathrm{f}(97)=\mathrm{f}(1+\mathrm{f}(99))=\mathrm{f}(107)=109$;
$\mathrm{f}(99)=\mathrm{f}(1+\mathrm{f}(101))=\mathrm{f}(1+103)=\mathrm{f}(104)=106$;
Se observă că plecând de la $x=99$, dacă $x$ scade cu $2, f(x)$ crește cu 3 .
Plecând de la 99 avem $99-59=40 ; 40 / 2=20 ; 20 * 3=60 ; 60+106=166$ sau
Plecând de la 95 avem $95-59=36 ; 36 / 2=18 ; ~ 18 * 3=54 ; 54+112=166$
14. Răspuns corect: e) Cel mai mic număr de operații s-a efectuat pentru $\mathbf{z}$.

Indicații: ca operații avem:
operații comune la toate tablourile:
-numărul de comparații pentru determinarea minimului este 6
-inițializarea indicelui valorii minime 3 operații
-verificare dacă minimul se află pe poziția i (ca să nu fac interschimbare cu el însuși)
3 operații
diferente:
-(d1) nr de interschimbări
-(d2) nr de actualizări ale indicelui minimului
v: d1=3 și d2=3
x: d1=2 și d2=2
y: d1=2 și d2=2
z: d1=1 și d2=2
15. Răspuns corect: d) divizorii primi ai lui $\mathbf{x}$ și numărul tuturor divizorilor lui $\mathbf{x}$

EXEMPLU: pentru x=36 se afișează 239
Indicații: este o descompunere în factori primi care afișează divizorii primi (la prima apariție în descompunere ) și returnează numărul total de divizori calculat ca produs de exponenți plus 1.
$36=2^{2} \cdot 3^{2}$
numărul divizorilor este $(2+1) \cdot(2+1)=9$

\section*{Varianta 25}

\section*{Indicații și răspunsuri}
1. Răspuns corect: b) 4 *a* (a-1) <a*a-2

Indicații: $4 * a *(a-1)<a * a-2$ se scrie astfel $3 a^{2}-4 a+2<0$ unde delta este -8 , rezultă că ecuația de gradul al II-lea va fi pozitivă mereu.
2. Răspuns corect: a) 167238945

Indicații:
\begin{tabular}{|l|l|l|l|}
\hline & & & Se afișează \\
\hline & i=4 & j=0 & A [4] [0]=1 \\
\hline j<>4 (A) & $i=3$ & j=0 & A[3] [0] $=6$ \\
\hline & $i=3$ & j=1 & A[3][1] $=7$ \\
\hline j<>4 (A) & i=2 & j=1 & A[2][1] $=2$ \\
\hline & i=2 & j=2 & A[2][2] $=3$ \\
\hline j<>4 (A) & $\mathrm{i}=1$ & j=2 & A[1][2]=8 \\
\hline & $i=1$ & j=3 & A [1] [3] $=9$ \\
\hline j<>4 (A) & $i=0$ & j=3 & A[0][3]=4 \\
\hline & $i=0$ & j=4 & A[0][4] $=5$ \\
\hline j<>4 (F) & & & \\
\hline
\end{tabular}
3. Răspuns corect: e) 7

Indicații:

Ad (3)
![](https://cdn.mathpix.com/cropped/2025_04_17_46e04c6acd873ea9558dg-290.jpg?height=170&width=413&top_left_y=1634&top_left_x=620)

Ad (7)
![](https://cdn.mathpix.com/cropped/2025_04_17_46e04c6acd873ea9558dg-290.jpg?height=69&width=285&top_left_y=1825&top_left_x=741)

Ad (9)
![](https://cdn.mathpix.com/cropped/2025_04_17_46e04c6acd873ea9558dg-290.jpg?height=71&width=285&top_left_y=1919&top_left_x=741)

El()
![](https://cdn.mathpix.com/cropped/2025_04_17_46e04c6acd873ea9558dg-290.jpg?height=69&width=286&top_left_y=1635&top_left_x=1299)

Ad (5)
![](https://cdn.mathpix.com/cropped/2025_04_17_46e04c6acd873ea9558dg-290.jpg?height=74&width=286&top_left_y=1730&top_left_x=1299)

Ad (2)
![](https://cdn.mathpix.com/cropped/2025_04_17_46e04c6acd873ea9558dg-290.jpg?height=85&width=280&top_left_y=1828&top_left_x=1302)

El()
![](https://cdn.mathpix.com/cropped/2025_04_17_46e04c6acd873ea9558dg-290.jpg?height=68&width=283&top_left_y=1923&top_left_x=1303)
4. Răspuns corect: c) 2

Indicații:
![](https://cdn.mathpix.com/cropped/2025_04_17_46e04c6acd873ea9558dg-290.jpg?height=232&width=547&top_left_y=2047&top_left_x=588)

Componenta conexă I este formată din nodurile: 1, 4, 6
Componenta conexă II este formată din nodurile: 2, 3, 5
5. Răspuns corect: f) 94

Indicații:
![](https://cdn.mathpix.com/cropped/2025_04_17_46e04c6acd873ea9558dg-291.jpg?height=567&width=405&top_left_y=378&top_left_x=361)
6. Răspuns corect: e) 8

Indicații: 114, 123, 132, 141, 213, 222, 231, 312
7. Răspuns corect: a) xxmmnn

Indicații: Instrucțiunea repetitivă for parcurge șirul "examen" până la penultimul caracter inclusiv și verifică dacă valoarea elementului de pe poziția i este strict mai mică din punct de vedere lexicografic decât valoarea elementului următor, $a[i+1]$. Dacă acest element este strict mai mic decât următorul, lui $a$ [i] i se atribuie valoarea lui $a[i+1]$.
8. Răspuns corect: b) 4324

Indicații: Deoarece există $o$ instrucțiune if cu condiția $\mathbf{n}<=100$, înseamnă că funcția numar se va opri când $\mathbf{n = 8 2}$. Până atunci va verifica fiecare cifră a numărului n . Dacă cifra este mai mică decât 5 aceasta va fi afișată imediat deoarece afișarea se face înainte de reapelarea funcției (4, 3, 2 și 4). Dacă cifra este mai mare decât 5 ( 9 și 7) aceasta va fi afișată la sfârșit, în ordine inversă a găsirii ei, deoarece afișarea se execută după reapelarea funcției.
9. Răspuns corect: b) 3

Indicații: Liniile de cod află câți divizori primi are valoarea memorată în variabila a. Variabila c, care inițial are valoarea 2 , este incrementată cu 1 până când valoarea variabilei a devine 1. Valoarea variabilei a ajunge la 1 deoarece este împărțită, pe rând, dacă este posibil, la valorile luate de variabila c. De câte ori se găsește un număr în variabila c care divide valoarea variabilei $a$, valoarea variabilei b crește cu 1 .
10. Răspuns corect: $c$ ) $v=[5,8,2,6,6,5,4,4]$

Indicații:
\begin{tabular}{|l|ll|l|ll|l|}
\hline $\mathrm{i}=0$ & $\mathrm{v}[0]<5$ & (F) & $\mathrm{v}[0]=5$ & $\mathrm{v}[7]>\mathrm{v}[0]$ & (A) & $\mathrm{v}[7]=4$ \\
\hline $\mathrm{i}=1$ & $\mathrm{v}[1]<5$ & (F) & $\mathrm{v}[1]=8$ & $\mathrm{v}[6]>\mathrm{v}[1]$ & (F) & $\mathrm{v}[6]=4$ \\
\hline $\mathrm{i}=2$ & $\mathrm{v}[2]<5$ & (A) & $\mathrm{v}[2]=2$ & $\mathrm{v}[5]>\mathrm{v}[2]$ & (A) & $\mathrm{v}[5]=5$ \\
\hline $\mathrm{i}=3$ & $\mathrm{v}[3]<5$ & (A) & $\mathrm{v}[3]=6$ & $\mathrm{v}[4]>\mathrm{v}[3]$ & (F) & $\mathrm{v}[4]=6$ \\
\hline
\end{tabular}
11. Răspuns corect: d) $\frac{c \cdot(c+2)}{4}$

Indicații: $2+4+6+\cdots+n=2 *\left(1+2+3+\cdots+\frac{n}{2}\right)=2 * \frac{\frac{n}{2} *\left(\frac{n}{2}+1\right)}{2}$.
Instrucțiunea repetitivă for calculează suma $1+2+3+\cdots+\frac{n}{2}$.
12. Răspuns corect: a) 1234567

Indicații:
\begin{tabular}{|l|l|l|}
\hline$i=1$ & $a[1]=7$ & $a[7]=1$ \\
\hline$i=2$ & $a[2]=6$ & $a[6]=2$ \\
\hline$i=3$ & $a[3]=5$ & $a[5]=3$ \\
\hline
\end{tabular}
\begin{tabular}{|l|l|l|}
\hline$i=4$ & $a[4]=4$ & $a[4]=4$ \\
\hline$i=5$ & $a[5]=3$ & $a[3]=5$ \\
\hline$i=6$ & $a[6]=2$ & $a[2]=6$ \\
\hline$i=7$ & $a[7]=1$ & $a[1]=7$ \\
\hline
\end{tabular}
13. Răspuns corect: d) 101

Indicații: Instrucțiunile respective numără căte cifre impare există, în total, în intervalul $[200,300]$. Prin urmare pe poziția cifrei sutelor, o cifră impară va apărea doar pentru numărul 300.
Pe poziția zecilor, o cifră impară, se va regăsi de exact 50 de ori în acest interval, pentru numerele de forma $\overline{2 a b}$ unde $a \in\{\mathbf{1}, \mathbf{3}, \mathbf{5}, \mathbf{7}, \mathbf{9}\}$ iar $b \in\{0,1,2, \ldots, 9\}$.
Cifra unităților va avea o valoare impară pentru numere de forma $\overline{2 a b}$ unde $a \in$ $\{0,1,2, \ldots, 9\}$ iar $b \in\{1,3,5,7,9\}$. Putem deduce de aici că în intervalul [200,300] pe poziția cifrei unităților vom regăsi o cifră impară de exact 50 de ori.
Pe poziția sutelor există o singură cifră impară. Deci $50+50+1=101$.
14. Răspuns corect: $c$ ) determinantul matricei

Indicații: Subprogramul mat calculează recursiv determinantul matricei d, primită ca parametru, prin descompunerea acesteia după linii și coloane.
În variabila e, declarată tot ca tablou bidimensional, este reținută matricea rezultată după descompunerea matricei inițiale, în funcție de elementul d[i] [j]. Prin urmare, tabloul bidimensional e va reține o matrice de n -1 linii și n -1 coloane.
Aceasta metodă se aplică recursiv până când matricea reținută în tabloul bidimensional e va avea un singur element.
15. Răspuns corect: b) 5417032963258410

Indicații: În cazul de față funcția recursivă denumită afis se va opri în momentul în care $\mathbf{k}$ devine $\mathbf{1}$. De menționat că, în cazul de față, $\mathbf{k}$ pornește de la valoarea 8 care este valoarea sumei indicilor lui a [4] [4], elementul din dreapta jos.
Valoarea variabilei $\mathbf{k}$ scade cu 1 la fiecare reapelare a funcției. Pentru $\mathbf{k}$ par se vor afișa elementele de la stânga spre dreapta iar pentru $\mathbf{k}$ impar se vor afișa elementele de la dreapta spre stânga.
$$
v=\left(\begin{array}{llll}
v_{11} & v_{12} & v_{13} & v_{14} \\
v_{21} & v_{22} & v_{23} & v_{24} \\
v_{31} & v_{32} & v_{33} & v_{84} \\
v_{41} & v_{42} & v_{43} & v_{44}
\end{array}\right) \quad V=\left(\begin{array}{llll}
0 & 1 & 2 & 3 \\
4 & 5 & 6 & 7 \\
8 & 9 & 0 & 1 \\
2 & 3 & 4 & 5
\end{array}\right)
$$

\section*{Varianta 26}

\section*{Indicații și răspunsuri}
1. Răspuns corect: e) $C++:(a \% 3+a \% 7) / 9$

Pascal: (a MOD 3+a MOD 7) DIV 9
Indicații: Restul împărțirii unui număr la 3 poate fi 0,1 sau 2. Restul împărțirii unui număr la 7 poate fi maxim 6. În concluzie, suma acestor două resturi nu poate fi mai mare de 8.
2. Răspuns corect: b) $v=(5,3,4,8,6,2,1,9)$

Indicații: Instrucțiunile din cadrul instrucțiunii repetitive while (C++: v[i]=v[i]+v[j]; v[j]=v[i]-v[j]; v[i]=v[i]-v[j]; sau Pascal: v[i]:=v[i]+v[j]; v[j]:=v[i]-v[j]; v[i]:=v[i]-v[j];) interschimbă valoarea aflată în $\mathbf{v}[i]$ cu valoarea aflată în $v[j]$.
\begin{tabular}{|c|c|c|c|}
\hline$i=0$ & $j=1$ & $v[0]=5$ & $v[1]=3$ \\
\hline$i=2$ & $j=3$ & $v[2]=4$ & $v[3]=8$ \\
\hline$i=4$ & $j=5$ & $v[4]=6$ & $v[5]=2$ \\
\hline$i=6$ & $j=7$ & $v[6]=1$ & $v[7]=9$ \\
\hline
\end{tabular}
3. Răspuns corect: f) poLItEHnica

Indicații: Toate literele mai mici decât litera n, în ordine alfabetică, până la poziția 7 din șirul de caractere memorat în variabila a, sunt transformate în litere mari (spre exemplu: 1 devine L )
4. Răspuns corect: d) $A=\left(\begin{array}{llll}0 & 1 & 2 & 3 \\ 1 & 2 & 2 & 4 \\ 2 & 2 & 4 & 3 \\ 3 & 4 & 3 & 6\end{array}\right)$

Indicații: Cele două instrucțiuni repetitive while parcurg toate elementele tabloului bidimensional A. Elementele care respectă condiția instrucțiunii if (i+j să fie par) sunt: $A[0][0]=0, A[0][2]=2, A[1][1]=2, A[1][3]=4, A[2][0]=2$, $A[2][2]=4, A[3][1]=4$ și $A[3][3]=6$.
Celelalte elemente au valoarea celui mai mare dintre indicii i și $\mathbf{j}: \mathrm{A}[0][1]=1$, $\mathbf{A}[0][3]=3, \quad \mathbf{A}[1][0]=1, \quad \mathbf{A}[1][2]=2, \quad \mathbf{A}[2][1]=2, \quad \mathbf{A}[2][3]=3$, A[3][0]=3 și A[3][2]=3.
5. Răspuns corect: d) 11

Indicații:
\begin{tabular}{|l|l|l|l|l|l|}
\hline Ad (3) & Ad (7) & Ad (5) & El () & El () & Ad (8) \\
\hline $\square$ & & 5 & $\square$ & $\square$ & $\square$ \\
\hline & 7 & 7 & 7 & & 8 \\
\hline 3 & 3 & 3 & 3 & 3 & 3 \\
\hline
\end{tabular}
6. Răspuns corect: d) $\mathbf{2 4}$

Indicații: 4a8b (4185, 4189, 4581, 4589, 4981, 4985) - 6 numere 8a4b-6 numere
$$
\begin{aligned}
& \text { a4b8-6 numere } \\
& \text { a8b4-6 numere }
\end{aligned}
$$
7. Răspuns corect: f) 222112

Indicații: Nodul cu numărul 2 din varianta de răspuns a indică faptul că este adiacent cu $\mathbf{5}$ noduri chiar dacă gradul nodului $\mathbf{3}$ este $\mathbf{0}$. Același raționament se aplică și pentru varianta d. Într-un graf neorientat suma gradelor tuturor nodurilor trebuie să fie un număr par. Prin urmare variantele b și e sunt false. Nodul cu numărul $\mathbf{3}$ din varianta de răspuns c are gradul 7 chiar dacă graful are $\mathbf{6}$ noduri.
8. Răspuns corect: d) $\boldsymbol{T}=\left(\begin{array}{l}05731312)\end{array}\right.$

\section*{Indicații:}

Reprezentarea grafică a variantei a:
![](https://cdn.mathpix.com/cropped/2025_04_17_46e04c6acd873ea9558dg-294.jpg?height=366&width=372&top_left_y=809&top_left_x=508)

Reprezentarea grafică a variantei $\mathbf{c}$ :
![](https://cdn.mathpix.com/cropped/2025_04_17_46e04c6acd873ea9558dg-294.jpg?height=367&width=372&top_left_y=1242&top_left_x=497)

Reprezentarea grafică a variantei e:
![](https://cdn.mathpix.com/cropped/2025_04_17_46e04c6acd873ea9558dg-294.jpg?height=367&width=372&top_left_y=1676&top_left_x=497)

Reprezentarea grafică a variantei $\mathbf{b}$ :
(1)
![](https://cdn.mathpix.com/cropped/2025_04_17_46e04c6acd873ea9558dg-294.jpg?height=272&width=372&top_left_y=894&top_left_x=1207)

Reprezentarea grafică a variantei $\mathbf{d}$ :
![](https://cdn.mathpix.com/cropped/2025_04_17_46e04c6acd873ea9558dg-294.jpg?height=367&width=356&top_left_y=1242&top_left_x=1291)

Reprezentarea grafică a variantei $\mathbf{f}$ :
![](https://cdn.mathpix.com/cropped/2025_04_17_46e04c6acd873ea9558dg-294.jpg?height=373&width=362&top_left_y=1700&top_left_x=1250)
9. Răspuns corect: b) $a^{2}+1$

Indicații: Cele două instrucțiuni repetitive while și do... while / repeat... until aflate în interiorul instrucțiunii for parcurg intervalul [1,a]. Instrucțiunea while parcurge intervalul [1,i] iar instrucțiunea do... while / repeat... until parcurge intervalul complementar [i+1,a].
Deoarece aceste instrucțiuni se află într-un for care parcurge același interval [1, a], variabila $\mathbf{s}$ va avea valoarea $\mathbf{a}^{2}$. Va afișa $\mathbf{a}^{2}+1$ deoarece instrucțiunea do . . . while
/ repeat. . . until va incrementa cu 1 variabila s în ultima apelare a instrucțiunii chiar dacă condiția este falsă deoarece este instrucțiune repetitivă cu test final.
Pentru $\mathbf{C}++$ nu se va afișa $\mathbf{a}^{2}$, ci $\mathbf{a}^{2}+1$ deoarece $\mathbf{s}--$ decrementează variabila $\mathbf{s}$ după afișare!
10. Răspuns corect: d) 53078520

Indicații: În cazul de față funcția recursivă denumită afis se va opri în momentul în care $\mathbf{k}$ devine $\mathbf{1}$. De menționat că, în cazul de față, $\mathbf{k}$ pornește de la valoarea 8 care este valoarea sumei indicilor lui a [4] [4], elementul din dreapta jos.
Valoarea variabilei $\mathbf{k}$ scade cu 2 la fiecare reapelare a funcției. Elementele se vor afișa din partea dreaptă spre cea stângă.
$$
\boldsymbol{v}=\left(\begin{array}{llll}
v_{11} & v_{12} & v_{13} & v_{14} \\
v_{21} & v_{22} & v_{23} & v_{k 4} \\
v_{31}^{2} & v_{32} & v_{33} & v_{34} \\
v_{41} & v_{42} & v_{43} & v_{44}
\end{array}\right) \quad V=\left(\begin{array}{cccc}
0 & 1 & 2 & 3 \\
4 & 5 & 6 & 7 \\
8 & 9 & 0 & 1 \\
2 & 3 & 4 & 5
\end{array}\right)
$$
11. Răspuns corect: f) $\underbrace{\boldsymbol{f}(\boldsymbol{t}) \circ \ldots \circ f(t)}_{c}$

Indicații: abc (functie (t) , c-1) apelează subprogramul functie () de c ori. Această apelare se face cu rezultatul primit deja de la subprogramul functie(), adică $\mathbf{f}(\mathbf{f}(\mathbf{x})$ ).
12. Răspuns corect: d) cuei

Indicații:
Pentru C++: A fost definit un tablou bidimensional de caractere, astfel:
\begin{tabular}{|c|c|c|c|c|c|c|c|c|c|c|c|}
\hline & $\mathbf{0}$ & $\mathbf{1}$ & $\mathbf{2}$ & $\mathbf{3}$ & $\mathbf{4}$ & $\mathbf{5}$ & $\mathbf{6}$ & $\mathbf{7}$ & $\mathbf{8}$ & $\mathbf{9}$ & $\mathbf{1 0}$ \\
\hline $\mathbf{1}$ & b & a & c & a & L & a & u & r & e & a & t \\
\hline $\mathbf{2}$ & l & i & c & e & U & & & & & & \\
\hline $\mathbf{3}$ & e & x & a & m & E & n & e & & & & \\
\hline $\mathbf{4}$ & p & $\circ$ & l & i & T & e & h & n & i & c & a \\
\hline
\end{tabular}

Prin urmare sunt afişate elementele:
\begin{tabular}{|l|l|}
\hline$i=1$ & $a[1][2]=` c^{`}$ \\
\hline$i=2$ & $a[2][4]=`{ }^{\prime}$ \\
\hline$i=3$ & $a[3][6]=` e^{`}$ \\
\hline$i=4$ & $a[4][8]=` i$ \\
\hline
\end{tabular}

Pentru Pascal: În comparație cu C++ unde șirul de caractere pornește de la 0 , aici pornește de la 1. Prin urmare avem următoarele date:
\begin{tabular}{|l|l|l|l|l|l|l|l|l|l|l|l|}
\hline & 1 & 2 & 3 & 4 & 5 & 6 & 7 & 8 & 9 & 10 & 11 \\
\hline 1 & b & a & c & a & L & a & u & R & e & a & T \\
\hline 2 & 1 & i & c & e & U & & & & & & \\
\hline 3 & e & x & a & m & E & n & e & & & & \\
\hline 4 & p & $\bigcirc$ & 1 & i & T & e & h & N & i & c & a \\
\hline
\end{tabular}

Următoarele elemente sunt afișate:
\begin{tabular}{|l|l|}
\hline$i=1$ & $a[1][3]=` c `$ \\
\hline$i=2$ & $a[2][5]=` u$ \\
\hline$i=3$ & $a[3][7]=` e `$ \\
\hline$i=4$ & $a[4][9]=` i `$ \\
\hline
\end{tabular}
13. Răspuns corect: a) $17 \mathbf{5}$

Indicații: $m=6, n=3$ rezultă că după apelarea $f 1(m, n)$ vor fi următoarele valori:
$\mathrm{f} 1(6,3)=10, \mathrm{~m}=6, \mathrm{n}=2$. Apelarea $\mathrm{f} 1(\mathrm{f} 1(\mathrm{~m}, \mathrm{n}), \mathrm{m})$ se face, de fapt, pentru $\mathrm{f} 1(10,6)$ care conduce la valorile $\mathrm{f} 1(10,6)=17, \mathrm{~m}=5, \mathrm{n}=2$.
14. Răspuns corect: $e$ ) $b=\left(\begin{array}{lll}29 & 38 & 47 \\ 38 & 50 & 62 \\ 47 & 62 & 77\end{array}\right)$

\section*{Indicații:}
\begin{tabular}{|c|c|c|c|}
\hline$i=1$ & $j=1$ & $k=1$ & $b[1][1]=4$ \\
\hline$i=1$ & $j=1$ & $k=2$ & $b[1][1]=13$ \\
\hline$i=1$ & $j=1$ & $k=3$ & $b[1][1]=29$ \\
\hline$i=1$ & $j=2$ & $k=1$ & $b[1][2]=6$ \\
\hline$i=1$ & $j=2$ & $k=2$ & $b[1][2]=18$ \\
\hline$i=1$ & $j=2$ & $k=3$ & $b[1][2]=38$ \\
\hline$i=1$ & $j=3$ & $k=1$ & $b[1][3]=8$ \\
\hline$i=1$ & $j=3$ & $k=2$ & $b[1][3]=23$ \\
\hline$i=1$ & $j=3$ & $k=3$ & $b[1][3]=47$ \\
\hline
\end{tabular}

Identic se procedează și pentru $i=2$ și $i=3$.
15. Răspuns corect: e) $\mathbf{2 4}$

Indicații: $\mathbf{n}=\mathbf{4}$ rezultă că variabila b din functie devine 12 . Dacă b devine 12 rezultă că și $\mathbf{n}$ devine 12. Instrucțiunea $a=2 * a$; atribuie lui $a$ valoarea $2 * n$ iar $n$ este 12 , adică 24.

\section*{Varianta 27}

\section*{Indicații și răspunsuri}
1. Răspuns corect: b) -24

Indicații: Se evaluează mai întâi rezultatul din paranteză, care este -2, apoi se calculează câtul împărțirii lui 16 la -2 și în final acesta se înmulțește cu 3.
2. Răspuns corect: b) $\mathrm{n} / 10$ \% 10 în varianta $\mathrm{C} / \mathrm{C}++$, respectiv n div $10 \bmod$ 10 în varianta Pascal.
EXEMPLU: Pentru $\mathrm{n}=3185, \mathrm{n} / 10=\mathrm{n}$ div $10=318$, iar $\mathrm{n} / 10 \% 10=$ $318 \% 10=8$, respectivn div $10 \bmod 10=318 \bmod 10=8$.
3. Răspuns corect: f) La final d1 și d2 vor fi egale doar dacă $\mathbf{n}$ reține un număr pătrat perfect.
EXEMPLE: Pentru $\mathrm{n}=11$ vom avea $\mathrm{d} 1=1$ și $\mathrm{d} 2=11$, deci $\mathrm{d} 1 \neq \mathrm{d} 2$, pentru $\mathrm{n}=125=5^{3}$ vom avea d1=5 și d2=25, deci d1 $=\mathrm{d} 2$, pentru $\mathrm{n}=21$ vom avea $\mathrm{d} 1=3$ și d2=7, deci $d 1 \neq d 2$. Pentru orice pătrat perfect de forma $n=r^{2}$, vom avea $d 1=d 2=r$. De exemplu, pentru $\mathrm{n}=36$ vom avea $\mathrm{d} 1=\mathrm{d} 2=6$.
Indicații: Algoritmul reține la final în d1 cel mai mare divizor al lui $\mathbf{n}$ al cărui pătrat e mai mic sau egal cu n, iar în d2 divizorul „complementar", care înmulțit cu d1 dă rezultatul n . Doar în cazul pătratelor perfecte $\mathrm{d} 1=\mathrm{d} 2$.
4. Răspuns corect: a) for ( $\mathrm{j}=\mathrm{n}-2$; j >= $\mathbf{i}$; $\mathrm{j}-\mathrm{-}$ ) în varianta $\mathrm{C} / \mathrm{C}++$, respectiv for $j$ := $\mathrm{n}-2$ downto i do în varianta Pascal.
EXEMPLU: Dacă $n=\mathbf{4}$ și $v=(3,2,1,4)$, la primul pas, în care $i=0$ vor fi comparate 1 și 4, fără a fi nevoie de interschimbare, apoi 2 și 1, care vor fi interschimbate și la final 3 și 1, care vor fi de asemenea interschimbate. Tabloul va deveni astfel ( $1,3,2,4$ ). Pentru i=1 vor fi comparate 2 și 4 fără a fi interschimbate, apoi 3 și 2, care vor fi interschimbate. Tabloul va deveni (1,2,3,4). Pentru i=2 vor mai fi comparate, fără a fi interschimbate 3 și 4.
Indicații: Algoritmul se aseamănă foarte mult cu sortarea prin metoda bulelor (Bubble Sort), constând de asemenea din parcurgeri succesive ale tabloului, în cadrul cărora sunt comparate și eventual interschimbate elemente aflate pe poziții consecutive. În cazul său însă parcurgerile se fac de la dreapta la stânga, în cadrul fiecăreia elementul minim fiind mutat în stânga subsecvenței prelucrate.
5. Răspuns corect: $c) f(1, n)==n$ în varianta $C / C++$, respectiv $f(1, n)=n$ în varianta Pascal.
Indicații: Apelul $\mathbf{f}(\mathbf{d}, \mathrm{n})$ returnează cel mai mic divizor al lui $\mathbf{n}$ mai mare strict ca d. În cazul în care $\mathbf{n}$ este prim acesta va fi $\mathbf{n}$ și reciproc (în enunț se garantează că $\mathbf{n}$ este natural, mai mare strict ca 1).
6. Răspuns corect: e) $\mathbf{3 6}$

Indicații: Un arbore cu 10 vârfuri are cu siguranță 9 muchii, iar un graf complet cu 10 vârfuri are cu siguranță 10* (10-1) /2 = 45 de muchii. Prin urmare, vor trebui adăugate 45-9 = 36 de muchii.
7. Răspuns corect: e) 65

Indicații: Secvența de instrucțiuni construiește următoarea matrice:
\begin{tabular}{rrrrr}
25 & 24 & 23 & 22 & 21 \\
20 & 19 & 18 & 17 & 16 \\
15 & 14 & 13 & 12 & 11 \\
10 & 9 & 8 & 7 & 6 \\
5 & 4 & 3 & 2 & 1
\end{tabular}
8. Răspuns corect: c) $(2,1,0,2,0)$

Indicații: Dacă numerotăm vârfurile ca în figura alăturată, vom obține șirul gradelor interne de la punctul c)
![](https://cdn.mathpix.com/cropped/2025_04_17_46e04c6acd873ea9558dg-298.jpg?height=329&width=649&top_left_y=481&top_left_x=1101)
9. Răspuns corect: c) Algoritmul generează în ordine lexicografică anagramele cuvântului citit care nu au vocale pe poziții alăturate.
Indicații: Ordinea alfabetică a literelor din cuvintele pilo și poli este i, l, o, p. Prin urmare varianta a) nu are cum să fie cea corectă (este înaintea lui pilo în ordine invers lexicografică). Variantele b) și d) nu sunt corecte, deoarece în ambele cazuri, penultimul cuvânt generat ar fi poil.
10. Răspuns corect: a) strcpy ( $t$, strchr (s, ' ' )) ; în varianta C/C++, respectiv t:=copy (s, pos(' ', s), 5) ; în varianta Pascal.
Indicații: Variantele b), d), e) și f) ar produce erori de compilare, iar c) ar face ca în variabila $t$ să fie reținut "Politehnica 20202020".
11. Răspuns corect: b) 31

Indicații: Ordinea apelurilor recursive va fi: $\mathrm{f}(24,34)=2+f(25,34)=$ $2+(1+f(26,33))=3+f(26,31)=3+f(26,29)=3+(1+f(27,28))$ $=4+f(27,26)=4+27=31$
12. Răspuns corect: f) 3

Indicații: Putem elimina muchiile $\{1,3\},\{3,4\}$ și $\{3,5\}$.Vom avea 3 cicluri elementare: $(1,5,6,1),(2,5,6,2)$ și $(1,5,2,6,1)$.
13. Răspuns corect: d) 011135

Indicații: Primii trei vectori în ordine lexicografică sunt: $(0,1,1,1,2,5)$, $(0,1,1,1,3,5)$ și $(0,1,1,1,4,5)$. Vectorul $(0,1,2,3,1,1)$ nu este al doilea, $(0,1,1,1,1,3)$ corespunde unui arbore cu înălțimea 2 , iar $(0,1,1,1,2,6)$ nu corespunde unui arbore ( 6 nu poate fi propriul său tată).
14. Răspuns corect: b) $m$ * $m>=x$

EXEMPLU: Dacă $\mathbf{x}=30$ și apelurile subprogramului sunt următoarele (făcând observația că rezultatul dorit nu poate fi mai mare ca $\mathbf{x}$ ): $\operatorname{rad}(1,30,30)=\operatorname{rad}$ $(1,15,30)=\operatorname{rad}(1,8,30)=\operatorname{rad}(5,8,30)=\operatorname{rad}(6,8,30)=$ $\operatorname{rad}(6,7,30)=\operatorname{rad}(6,6,30)=6$
Indicații: Subprogramul folosește o versiune modificată a căutării binare pentru a obține rezultatul dorit. Astfel dacă elementul din mijlocul intervalului în care se caută rezultatul are proprietatea dorită (pătratul său e mai mare sau egal cu x), căutarea continuă în prima jumătate, altfel în cea de-a doua.
15. Răspuns corect: $d$ ) $s-=v[j]$; în varianta $C / C++$, respectiv $s:=s-v[j]$; în varianta Pascal.

EXEMPLU: Dacă $n=8$, v=(3,5,4,1,2,8,19,3) și $t=10$ atunci 1 max va avea la sfârșitul executării secvenței de instrucțiuni valoarea 3 . Mai precis pentru $i=0$ se va găsi $\mathbf{j}=0$, pentru $\mathbf{i = 1}$ se va găsi de asemenea $\mathbf{j}=0$, deoarece suma primelor $\mathbf{2}$ elemente ale tabloului nu depășește $t=10$, pentru $i=2$ se va găsi $j=1$, deoarece suma primelor 3 elemente depășește $t$, dar $v[1]+v[2] \leq t$, pentru $i=3$ se va găsi $j=1$, pentru $i=4$ se va găsi $j=2$, pentru $i=5$ se va găsi $j=4$, pentru $i=6$ se va găsi $j=7$, pentru $i=7$ se va găsi $j=7$,
Indicații: Pentru fiecare i cuprins între 0 și n-1, algoritmul determină în timp liniar cel mai mare $\mathbf{j}$ cu proprietatea că subsecvența de la poziția i până la poziția j inclusiv respectă restricția de a avea suma mai mică sau egală cu t. Pentru verificarea acestei condiții este păstrată în permanență în variabila s suma elementelor din subsecvența curentă. Incrementarea lui $j$ este echivalentă cu eliminarea lui $v[j]$ din subsecvență, prin urmarea este necesar ca s să scadă cu v[j].

\section*{Varianta 28}

\section*{Indicații și răspunsuri}
1. Răspuns corect: $d)(a+b+c+d) * 0.25$

Indicații: Varianta a) ar fi corectă doar dacă suma $a+b+c+d$ ar fi scrisă între paranteze. Variantele b) , c), e) și f) sunt incorecte din punct de vedere matematic, deși nu au erori de sintaxă.
2. Răspuns corect: c) Atât S1, cât și S2

Indicații: Ambele secvențe de instrucțiuni au ca efect obținerea în variabila $\mathbf{p}$ a primei cifre a numărului reținut inițial de $n$. În cadrul $\mathbf{S 1} \mathbf{p}$ reține mai întâi o copie a lui n , iar apoi, cât timp p are mai mult de o cifră se elimină ultima dintre acestea. În S2 se reține în mod repetat în $\mathbf{p}$ ultima cifră a lui $n$, apoi aceasta se elimină prin păstrarea în $n$ a câtului împărțirii întregi a lui $n$ la 10. În felul acesta ultima cifră eliminată este evident prima cifră a numărului reținut inițial de $n$, aceasta fiind valoarea finală a lui $\mathbf{p}$. Singura diferență dintre S1 și S2 este că în cazul celei de-a doua secvențe de instrucțiuni valoarea inițială a lui $\mathbf{n}$ se va pierde, aceasta fiind înlocuită cu 0 la sfârșitul executării buclei.
3. Răspuns corect: $f$ ) n \% $\mathrm{d}=\mathbf{0}$ în varianta $\mathrm{C} / \mathrm{C}++$, respectiv $\mathrm{n} \bmod \mathrm{d}=0$ în varianta Pascal.
EXEMPLU: Pentru $n=300$, când $d=2$, în bucla while ( $n \% d==0$ ) a programului C/C++, respectiv while n mod $\mathrm{d}=0$ do a variantei Pascal n va deveni 75 , iar $p$ și $n r$ vor deveni 1. Apoi, pentru $d=3$, în aceeași buclă $n$ va deveni $25, p$, din nou 1 , iar $n r$ va primi valoarea 2. Pentru $d=4 n u$ se va intra în bucla while internă, deci $p$ va rămâne 0 , iar $n$ și $n r$ nu se vor modifica. În fine, când $d=5, p$ va deveni din nou 1 , $n r$ va deveni 3 , iar $n$ va ajunge 1 , ceea ce va face ca bucla while exterioară să se încheie.
Indicații: Algoritmul găsește divizorii primi ai lui n bazându-se pe faptul că, dacă atunci când e găsit un divizor d, ne împărțit la el de câte ori e posibil, atunci nu se va intra în bucla interioară decât pentru divizorii primi ai numărului reținut inițial de $\mathbf{n}$.
4. Răspuns corect: e) $\mathrm{v}[\mathrm{i}+1]=\mathbf{x}$ în varianta $\mathrm{C} / \mathrm{C}++$, respectiv $\mathrm{v}[i+1]:=\mathbf{x}$ în varianta Pascal.
EXEMPLE: Dacă $n=5, v=(2,3,5,5,8)$ și $x=4$, la primul pas, în care $i=4$ sunt comparate 4 și 8, iar v[5] devine 8, apoi i devine 3 și, pentru că v[3]=5 și e în continuare mai mare ca $\mathbf{x}=4, \mathrm{v}[4]$ devine 5 . Același lucru se întâmplă pentru $i=2$ și $v$ [3] devine de asemenea 5. Bucla se încheie cu i=1, pentru că v[1]=3 și 3 nu e mai mare strict ca $\mathbf{x}=4$. Evident, 4 va fi inserat după 3 , pe poziția $i+1=2$. Vectorul va deveni astfel ( $2,3,4,5,5,8$ ).
Dacă $n=4, v=(2,3,5,5)$ și $x=1$, atunci în bucla while vor fi mutate cu o poziție spre dreapta toate elementele vectorului, iar la final i va avea valoarea -1. Pe poziția $i+1=0$ va fi scris $\mathbf{x}=1$. Vectorul va deveni astfel ( $1,2,3,5,5,8$ ).
Indicații: Algoritmul îl inserează pe $\mathbf{x}$ în $\mathbf{v}$ după ce mută spre dreapta cu o poziție toate componentele mai mari strict ca $\mathbf{x}$. Bucla while se poate termina fie atunci când toate elementele lui $\mathbf{v}$ sunt strict mai mici ca $\mathbf{x}$ și sunt mutate (i devine $\mathbf{- 1}$ ), fie când se
ajunge la un i cu proprietatea că $\mathbf{v}[\mathbf{i}] \leq \mathbf{x}$. În ambele cazuri este necesar ca $\mathbf{x}$ să fie inserat pe poziția $\mathbf{i}+1$.
5. Răspuns corect: b) 60 .

Indicații: Secvența de instrucțiuni construiește următoarea matrice:
\begin{tabular}{lllll}
5 & 4 & 3 & 2 & 1 \\
6 & 5 & 4 & 3 & 2 \\
7 & 6 & 5 & 4 & 3 \\
8 & 7 & 6 & 5 & 4 \\
9 & 8 & 7 & 6 & 5
\end{tabular}
6. Răspuns corect: b) $£(225)$

Indicații: În cadrul buclei while $r$ devine cel mai mic număr al cărui pătrat este mai mare sau egal cu $n$ (partea întreagă superioară a rădăcinii pătrate a lui $n$ ). Subprogramul returnează astfel 0 dacă și numai dacă $n$ este pătrat perfect.
7. Răspuns corect: d) (1,1,2,2)

Indicații: Variantele a) și f) sunt incorecte, pentru că un vârf nu poate avea gradul intern mai mare decât 3 . Varianta b) are suma gradelor interne mai mare decât suma gradelor externe, iar varianta e) are suma suma gradelor interne mai mică decât suma gradelor externe. În cazul variantei c) contradicția provine din faptul că vârful 4 are gradul intern 0 , deci nu poate exista un arc ( 1,4 ), iar 1 are gradul extern 3 . Graful cu arcele $(1,2),(1,3),(1,4),(2,1),(3,4),(4,3)$, corespunde șirului gradelor interne (1,1,2,2).
8. Răspuns corect: c) 5

Indicații: Arborele din figura alăturată corespunde vectorului de tați dat.
![](https://cdn.mathpix.com/cropped/2025_04_17_46e04c6acd873ea9558dg-301.jpg?height=537&width=752&top_left_y=1290&top_left_x=865)
9. Răspuns corect: a) strcpy ( $p, p+1$ ) în varianta $C / C++$, respectiv
delete ( $\mathrm{s}, \mathrm{p}, 1$ ) în varianta Pascal.
Indicații: Având în vedere că de fiecare dată caracterul $\mathbf{c}$ este căutat de la începutul șirului ( $\mathbf{p}=\mathbf{s t r c h r}(\mathbf{s}, \mathbf{c})$; , respectiv $p:=\operatorname{pos}(\mathbf{c}, \mathbf{s})$ ) de caractere $\mathbf{s}, \mathrm{e}$ necesar ca la fiecare iterație a buclei while să se șteargă ultima apariție găsită.
10. Răspuns corect: f) 012013

Indicații: Arborele apelurilor recursive este cel din figura de mai jos.
![](https://cdn.mathpix.com/cropped/2025_04_17_46e04c6acd873ea9558dg-302.jpg?height=432&width=936&top_left_y=245&top_left_x=356)

Apelul cu $n=3$ va afișa pe ecran 012, apoi apelul cu $n=1$ va afișa 01 și în final apelul cu $\mathrm{n}=3$ va mai afișa 3 . Celelalte 3 apeluri nu vor afișa nimic, întrucât condiția $\mathrm{n}>0 \mathrm{nu}$ va fi verificată în cazul lor.
11. Răspuns corect: b) Generarea tuturor permutărilor mulțimii $\{1,2, \ldots, n\}$.

Indicații: În cazul ambelor probleme elementul curent din vectorul soluție, sol [p] trebuie să respecte condiția sol [p] $\boldsymbol{\operatorname { s o n }}$ [i] pentru orice $\mathbf{i}<\mathrm{p}$.
12. Răspuns corect: d) G nu poate fi eulerian

Indicații: Un graf cu 10 vârfuri și mai puțin de 9 muchii nu poate fi conex, deci afirmația de la punctul a) este adevărată. Un graf care nu este conex nu poate fi hamiltonian, deci afirmația de la punctul c) este adevărată. Graful G poate avea două cicluri elementare cu câte 4 vârfuri și 4 muchii și 2 vârfuri izolate (4 componente conexe în total), deci afirmația de la punctul b) este adevărată. În fine, un graf eulerian poate avea vârfuri izolate, deci în cazul lui G putem avea un ciclu cu 8 vârfuri care să conțină toate cele 8 muchii și alte două vârfuri izolate. Prin urmare, afirmația de la punctul d) este singura care nu e adevărată.
13. Răspuns corect: e) 5

Indicațiii: $O$ soluție posibilă este un arbore cu vectorul de tați:
( $0,1,1,1,2,2,2,5,5,5,8,11$ ) cu lanțul ( $1,2,5,8,11,12$ ) de lungime 5 , care unește rădăcina 1 cu frunza 12.
14. Răspuns corect: a)
\end{verbatim}

if (v[n-2] > v[n-1])\\
\{\\[0pt]
aux = v[n-1];\\[0pt]
v[n-1] = v[n-2];\\[0pt]
v[n-2] = aux;\\
mysort(n - 1, v);\\
\}\\
pentru limbajul C/C++, respectiv\\[0pt]
if v[n-2] > v[n-1] then\\
begin\\[0pt]
aux := v[n-1];\\[0pt]
v[n-1] := v[n-2];\\[0pt]
v[n-2] := aux;\\
mysort(n-1, v)\\
end\\
pentru limbajul Pascal

\begin{verbatim}

Indicații: În urma apelului recursiv mysort ( $n-1, v$ ), primele $n-1$ componente ale lui $\mathbf{v}$ vor fi ordonate. Dacă $\mathbf{v}[\mathrm{n}-2] \leq \mathrm{v}[\mathrm{n}-1]$, atunci nu mai sunt necesare alte
prelucrări. În caz contrar ultimele două componente ale subsecvenței ( $v[0], v[1]$, $\ldots, v[n-2], v[n-1])$ care trebuie ordonată vor fi interschimbate, iar procesul va fi reluat pentru primele n -1 elemente printr-un nou apel recursiv cu aceleași argumente. Cum v[n-1] este cu siguranță elementul maxim după interschimbare, la revenirea din apelul recursiv, subsecvența va fi ordonată.
15. Răspuns corect: $c$ ) v [i]>s [m-1]

EXEMPLU: Dacă $n=8, \quad v=(3,3,1,8,2,1,5,4)$, corespunzător lui $X=33182154$ și $k=4$ atunci aplicând algoritmul vom obține vectorul $s=(8,2,5,4)$, corespunzător lui $Y=8254$. Acesta e cel mai mare număr natural care poate fi obținut din $\mathbf{X}$ prin eliminarea a exact $\mathbf{4}$ cifre, fără a schimba ordinea în care cifrele apăreau în X.

Indicații: Algoritmul folosește vectorul s, organizat după principiul „ultimul sosit primul servit". Componentele acestuia sunt acele cifre care pot face parte din Y. În momentul luării în considerare a unei noi cifre (v[i]) aceasta este introdusă cu siguranță în s(nu se știe încă nimic despre cifrele care urmează și deci acestea ar putea fi mai mici). Apariția cifrei curente (v [i]) poate avea ca efect eliminarea din sa altor cifre, mai mici, care nu trebuie să se afle înaintea lui v[i] în Y. Aceste eliminări se fac în bucla while prin decrementarea lui $m$ (numărul de elemente din vectorul $\mathbf{s}$ ).

\section*{Varianta 29}

\section*{Indicații și răspunsuri}
1. Răspuns corect: c) ( $n-1$ )/2

EXEMPLU: Pentru n=7 și tabloul unidimensional ( $\left.\begin{array}{lllll}4 & 2 & \underline{3} & 1 & \underline{6} \\ 8 & \text { 5 }\end{array}\right)$ se obține tabloul (5 861324 ).
Indicații: Elementul de pe poziția din mijloc nu este necesar să fie mutat. Se interschimbă 4 cu 5, 2 cu 8 și 3 cu 6, 3 interschimbări, adică se fac ( $n-1$ )/2 interschimbări.
2. Răspuns corect: b) (1 1510218 91)

Indicații: Se mută al treilea element spre stânga cu 2 poziții ș.a.m.d. Primele 2 elemente se mută în aceeași ordine la sfârș̦it.
3. EXEMPLU:
\begin{tabular}{|c|c|c|c|}
\hline & $\mathbf{x}$ & & \\
\hline $\mathbf{x}$ & & & $\mathbf{x}$ \\
\hline $\mathbf{x}$ & & & \\
\hline
\end{tabular}

Răspuns corect: c) 3142
Indicații: Două dame nu se atacă între ele dacă nu se află pe aceeași coloană, pe aceeași diagonală sau pe același rând. Astfel așezarea lor poate fi: dama 1 pe rândul 3, dama 2 pe rândul 1, dama 3 pe rândul 4, dama 4 pe rândul 2.
4. Răspuns corect: b) 10

EXEMPLU: Pentru tabloul unidimensional ( $\left.\begin{array}{llllllll}10 & 24 & 9 & 11 & 33 & 7 & 15\end{array}\right)$ se fac următoarele interschimbări:
Pasul 1: (10 91124715 33): 4 interschimbări
Pasul 2: (9 $\left.10 \begin{array}{llllll}9 & 11 & 15 & 24 & 33\end{array}\right): 3$ interschimbări
Pasul 3: (9 10 7111524 33): 1 interschimbare
Pasul 4: (9 710111524 33): 1 interschimbare
Pasul 5: ( $\begin{aligned} & 7 \\ & 9\end{aligned} 10111524$ 33): 1 interschimbare
Indicații: O interschimbare se face dacă sunt îndeplinite simultan comdițiile v[i]>v[j] și i<j.
5. Răspuns corect: f) 2678934567345683456934578

Indicații:
a) 456784567945689457894678956789 - lipsește numărul subliniat
b) $347893567835679 \quad 35689 \quad 36789 \quad 45678$ - lipsește numărul subliniat
c) 3457834569345683456726789 - nu sunt în ordine crescătoare
d) $134581345913467 \quad 1346913478 \quad 13479$ - lipsește numărul subliniat
e) $13458 \quad 1345913467 \quad 13468$ 13469-numărul subliniat are mai mult de 2 cifre alăturate de aceeași paritate
6. Răspuns corect: c) 2349

Indicații: Subprogramul parcurge recursiv cele două numere a și b și returnează un număr format cu cifra mai mare de pe aceeași poziție din cele două numere.
7. Răspuns corect: d) 3

Indicații: Numărul minim de comparații se obține folosind algoritmul de căutare binară. Comparația 1: elementul din mijloc: 73; Comparația 2: elementul din mijloc: 95; Comparația 3: elementul din mijloc: 82;
8. Răspuns corect: e) nedefinită

Indicații: Este o variabilă locală neiniţializată.
9. Răspuns corect: b) 190

Indicații: În total sunt 400 de elemente, 20 sunt pe diagonala secundară.
10. Răspuns corect: d) 1023

EXEMPLU: Fie cele 3 tije $\mathbf{a}$, $\mathbf{b}$ și c. Se mută discurile de pe tija a pe tija b, utilizând ca tijă intermediară tija c. Pentru $k=3$ discuri se fac 7 mutări: $a \rightarrow b, a \rightarrow c, b \rightarrow c, a \rightarrow b$, $c \rightarrow a, c \rightarrow b, a \rightarrow b$.
Indicații: Pentru $\mathbf{k}$ discuri este necesar un număr de $\mathbf{2}^{\mathbf{k}} \mathbf{- 1}$ mutări.
11. Răspuns corect: c) 8

Indicații: Graful nu conține bucle. Fiecare vârf este adiacent cu toate celelalte vârfuri. Dacă graful are n vârfuri există $\mathrm{n}(\mathrm{n}-1)$ arce. $56=\mathrm{n}(\mathrm{n}-1), \mathrm{n}=8$.
12. Răspuns corect: f) Doi dintre algoritmi nu diferă ca eficiență din punctul de vedere al timpului de executare.
Indicații:
a) Algoritmii $\mathrm{A}_{1}$ și $\mathrm{A}_{2}$ rezolvă problema pentru orice date de intrare cu valorile din interval.
b) Algoritmul $\mathrm{A}_{2}$ este cel mai eficient din punctul de vedere al timpului de executare. Are complexitatea $O(n)$ pentru parcurgerea șirului și determinarea sumei.
c) Algoritmul $\mathrm{A}_{4}$ are complexitate dată de o sortare rapidă, $\mathrm{O}(\mathrm{n} \log \mathrm{n})$.
d) Algoritmul $\mathrm{A}_{4}$ rezolvă problema.
e) Algoritmii rezolvă problema pentru orice date de intrare cu valorile din interval.
f) $\mathrm{A}_{3}$ și $\mathrm{A}_{4}$ au aceeași eficiență, complexitate dată de o sortare rapidă, $\mathrm{O}(\mathrm{n} \log \mathrm{n})$.

Observație: Există algoritmi care rezolvă această problemă într-un timp mai scurt (folosind operații pe biți).
13. Răspuns corect: f) $\mathrm{E}_{1}, \mathrm{E}_{2}$ și $\mathrm{E}_{3}$
14. Răspuns corect: c) 5040

EXEMPLU: Anagramele (nu neapărat în această ordine) sunt: aaccerrt, aaccertr, aaccetrr, ..., rtreccaa, trreccaa. În total 5040.
Indicații: Litera a apare de două ori, litera c apare de două ori, litera e apare o dată, litera $r$ apare de două ori, litera $t$ apare o dată; permutările sunt cu repetiție.
Numărul permutărilor este: 8 !/(2! $2!\cdot 1!\cdot 2!\cdot 1!)$
15. Răspuns corect: b) doar formula 1

Indicații:
Formula 1 poate fi obținută din ecuația
$\left(\begin{array}{ll}1 & 1 \\ 1 & 0\end{array}\right)^{n}=\left(\begin{array}{cc}F_{n+1} & F_{n} \\ F_{n} & F_{n-1}\end{array}\right)$
Claculăm determinantul: $(-1)^{n}=F_{n+1} * F_{n-1}-F_{n}^{2}$.
Știm că $M^{m} * M^{n}=M^{m+n}$, pentru orice matrice $M$ pătratică.
Așadar $(-1)^{m}=F_{m+1} * F_{m-1}-F_{m}^{2}$.
$M^{m}=\left(\begin{array}{cc}F_{m+1} & F_{m} \\ F_{m} & F_{m-1}\end{array}\right)$
$M^{n}=\left(\begin{array}{cc}F_{n+1} & F_{n} \\ F_{n} & F_{n-1}\end{array}\right)$
$$
\begin{aligned}
& M^{m+n}=\left(\begin{array}{cc}
F_{m+n+1} & F_{m+n} \\
F_{m+n} & F_{m+n-1}
\end{array}\right) \\
& M^{m} \cdot M^{m}=\left(\begin{array}{cc}
F_{m+1} & F_{m} \\
F_{m} & F_{m-1}
\end{array}\right) \cdot\left(\begin{array}{cc}
F_{n+1} & F_{n} \\
F_{n} & F_{n-1}
\end{array}\right)= \\
& \quad=\left(\begin{array}{cc}
F_{m+1} \cdot F_{n+1}+F_{m} \cdot F_{n} & F_{m+1} \cdot F_{n}+F_{m} \cdot F_{n-1} \\
F_{m} \cdot F_{n+1}+F_{m-1} \cdot F_{n} & F_{m} \cdot F_{n}+F_{m-1} \cdot F_{n-1}
\end{array}\right)=M^{m+n}= \\
& \quad=\left(\begin{array}{cc}
F_{m+n+1} & F_{m+n} \\
F_{m+n} & F_{m+n-1}
\end{array}\right)
\end{aligned}
$$

Identificăm:
$F_{m} \cdot F_{n+1}+F_{m-1} \cdot F_{n}=F_{m+n} \quad$ (a)
$F_{m} \cdot F_{n}+F_{m-1} \cdot F_{n-1}=F_{m+n-1} \quad$ (b)
Punem $m=n$
$$
\begin{equation*}
F_{n} \cdot F_{n+1}+F_{n-1} \cdot F_{n}=F_{2 n} \tag{a}
\end{equation*}
$$
$$
\begin{equation*}
F_{n}^{2}+F_{n-1}^{2}=F_{2 n-1} \tag{b}
\end{equation*}
$$

Din relația (a) rezultă
$F_{2 n}=F_{n} \cdot\left(F_{n+1}+F_{n-1}\right)=F_{n} \cdot\left(F_{n-1}+F_{n}+F_{n-1}\right)=F_{n} \cdot\left(2 \cdot F_{n-1}+F_{n}\right)$.
Așadar:
Dacă $n$ este par, înjumătățindu-l, rezultă:
$F(n)=F\left(\frac{n}{2}\right) \cdot\left[2 \cdot F\left(\frac{n}{2}-1\right)+F\left(\frac{n}{2}\right)\right]$.
Dacă $n$ este impar, din (b), rezultă:
$F(n)=\left[F\left(\frac{n+1}{2}\right)\right]^{2}+\left[F\left(\frac{n-1}{2}\right)\right]^{2}$.
Termeni, conform formulelor, sunt:
$1,1,2,3,5,8, \ldots$
Referitor la formulele 2 și 3, corect este: $F(n)=\frac{1}{\sqrt{5}} \cdot\left(\frac{1+\sqrt{5}}{2}\right)^{n}-\frac{1}{\sqrt{5}} \cdot\left(\frac{1-\sqrt{5}}{2}\right)^{n}$

\section*{Varianta 30}

\section*{Indicații și răspunsuri}
1. Răspuns corect: c$) \mathrm{n} / 2$

EXEMPLU: Pentru n=8 și tabloul unidimensional ( $\left.\begin{array}{llllllll}4 & 2 & \underline{3} & 1 & 9 & \underline{6} & 8 & 5\end{array}\right)$ se obține tabloul (5 8 ( 6
Indicații: Se interschimbă 4 cu 5, 2 cu 8,3 cu 6 și 1 cu 9,4 interschimbări, adică se fac n/2 interschimbări.
2. Răspuns corect: d) 0 elemente

Indicații: Nu este nevoie de spațiu de memorie suplimentar.
3. Răspuns corect: f) 0

Indicații: Nu există nicio soluție pentru 3 dame.
4. Răspuns corect: b) 35

Indicații: Matricea de adiacență are numărul de linii egal cu numărul de coloane.
5. Răspuns corect: e) dc $(x, y)=d c(y, x \bmod y)$

EXEMPLU: pentru $\mathbf{x = 6}$ și $\mathbf{y}=8$
a) dc $(6,8) \neq \operatorname{dc}(48,8)$
b) $\mathrm{dc}(6,8) \neq \mathrm{dc}(6,6)$
c) $\operatorname{dc}(6,8)=\operatorname{dc}(8,48)=\operatorname{dc}(48,384)=.$.
d) $\operatorname{dc}(6,8) \neq \operatorname{dc}(6,6)$
f) $\mathrm{dc}(6,8) \neq \mathrm{dc}(0,0)$

Indicații: Formula pentru a calcula cel mai mare divizor comun folosind algoritmul lui Euclid este dc $(\mathbf{x}, \mathrm{y})=\mathrm{dc}(\mathrm{y}, \mathbf{x} \bmod \mathbf{y})$. Se folosește recursiv până se obține restul 0.
6. Răspuns corect: b) 2 subtablouri, nu întotdeauna cu același număr de elemente. În partea stângă se vor găsi elementele mai mici decât pivotul, în partea dreaptă se vor găsi elementele mai mari decât pivotul, elementele egale cu pivotul pot rămâne în oricare parte a pivotului. Rezultă că cele două părți nu au întotdeauna același număr de elemente.
7. Răspuns corect: e) stivă

EXEMPLU: Se creează o stivă, operația de adăugare numită push(), memorând șirul de la primul caracter până la ultimul. Astfel fiecare caracter al șirului va fi, pe rând, în vârful stivei. Extragerea se face din vârful stivei, eliminând, pe rând, primul element, folosind operația numită pop().
Indicații: Stiva corespunde principiului LIFO (Last In First Out).
8. Răspuns corect: $d) f(n / 2)$; (respectiv $f(n \operatorname{div} 2)$ )

Indicații:
a) $\mathbf{f}(\mathrm{n}-2)$ subprogramul nu se încheie pentru valori impare ale lui n .
b) $\mathbf{f}(\mathrm{n}-1)$ subprogramul nu se încheie pentru valori negative ale lui n .
c) $\mathbf{f}(\mathrm{n} \% 2)$ (respectiv $\mathrm{f}(\mathrm{n} \bmod 2)$ ) subprogramul nu se încheie pentru valori impare ale lui $n$.
e) $f(n+2)$ subprogramul nu se încheie pentru valori ale lui $n$ diferite de $\mathbf{- 2}$ și 0 .
f) $f(n * 2)$ subprogramul nu se încheie pentru valori nenule ale lui $n$.
9. Răspuns corect: e) 4

Indicații: Un apel $\mathbf{f ( 1 )}$ are rezultatul 2, celălalt apel $\mathbf{f}(1)$ are rezultatul 2
10. Răspuns corect: d) 28

Indicații: Folosim principiul includerii-excluderii. Considerăm mulțimea $\mathrm{M}=\{\mathbf{1 0 1 , 1 0 2}$, ..., 200\}. Luăm alte trei submulțimi A, B, C care au ca elemente cele aflate în $\mathbf{M}$ divizibile cu 2, 3, respectiv 5. Determinăm elementele comune mulțimilor A și B, A și C, B și C, apoi comune mulțimilor A, B și C.
EXEMPLU: Pentru valorile date avem: cardM=100. cardA=50. cardB=33. cardC=20. $\operatorname{card}(\mathrm{AB})=17 . \operatorname{card}(\mathrm{AC})=10 . \operatorname{card}(\mathrm{BC})=7 . \operatorname{card}(\mathrm{ABC})=3$.
cardM-cardA-cardB-cardC+card(AB)+card(AC)+card(BC)-card(ABC) = 100-50-33-$20+17+10+7-3=28$.

\section*{11. Răspuns corect: a) $\mathbf{x x}$}

Indicații: Dacă programul generează permutări de elemente care se repetă atunci cele două caractere $\mathbf{x}$ nu sunt diferite; se formează o singură permutare.
12. Răspuns corect: d) 3

Indicații: sunt adevărate enunțurile 2, 3 și 5 .
Enunțul 1: se generează 28 de numere cu prima cifră 2.
Enunțul 2: 12457, 12459, 12479, 12679, 14679, 34679.
Enunțul 3: 13679 sau 24568.
Enunțul 4, un număr corect este 12789.
Enunțul 5: cifra 1 apare de 50 de ori pe prima poziție, cifra 9 apare de 50 de ori pe ultima poziție.
13. Răspuns corect: d) ( $\mathbf{n}-1$ ) !/2

Indicații: În graful neorientat complet orice permutare a celor $\mathbf{n}$ noduri este un ciclu hamiltonian. Un același ciclu, ca permutare circulară, se parcurge în $2 \cdot n$ moduri. Se pornește de la un nod în sensul acelor de ceasornic, dar și în sens trigonometric.
Exemplu: pentru $\mathrm{n}=4$,
Cicluri identice:
![](https://cdn.mathpix.com/cropped/2025_04_17_46e04c6acd873ea9558dg-308.jpg?height=52&width=1505&top_left_y=1508&top_left_x=375)
4 3), (4 $\left.1 \begin{array}{llll}4 & 2 & 3\end{array}\right),\left(\begin{array}{lllll}4 & 3 & 1 & 4\end{array}\right)$;
![](https://cdn.mathpix.com/cropped/2025_04_17_46e04c6acd873ea9558dg-308.jpg?height=52&width=1505&top_left_y=1611&top_left_x=375)
4 3), (4 $\left.2112 \begin{array}{l}4\end{array}\right),\left(\begin{array}{lllll}4 & 3 & 1 & 2\end{array}\right)$;
![](https://cdn.mathpix.com/cropped/2025_04_17_46e04c6acd873ea9558dg-308.jpg?height=52&width=1505&top_left_y=1714&top_left_x=375) 1 3), (4 $\left.1 \begin{array}{llll}4 & 3 & 2\end{array}\right)$, (4 $\left.2 \begin{array}{llll}4 & 1 & 4\end{array}\right)$.
În total avem 3 cicluri distincte hamiltoniene.
În general există $\mathrm{n}!/(2 \cdot n)=(n-1)!/ 2$ cicluri hamiltoniene distincte.
14. Răspuns corect: b) 5

Indicații: Numărul de grafuri orientate complete cu n noduri este $3^{\mathrm{C}_{\mathrm{n}}^{2}}$, adică $3^{\mathrm{n}(\mathrm{n}-1) / 2}$.
$3^{n(n-1) / 2}=59049,3^{n(n-1) / 2}=3^{10}, n=5$.
15. Răspuns corect: c) doar relațiile $E_{1}$ și $E_{2}$

Indicații:
$E_{1}$ : Se poate observa că fiecare al treilea termen din șirul lui Fibonacci este par.
$E_{2}: F_{n}=F_{n-1}+F_{n-2}=\left(F_{n-2}+F_{n-3}\right)+\left(F_{n-3}+F_{n-4}\right)=F_{n-2}+2 \cdot F_{n-3}++F_{n-4}=$ $\left(F_{n-3}+F_{n-4}\right)+2 \cdot F_{n-3}+F_{n-4}=3 \cdot F_{n-3}+F_{n-4}+\left(F_{n-5}+F_{n-6}\right)=3 \cdot F_{n-3}+$
$F_{n-3}+F_{n-6}=4 \cdot F_{n-3}+F_{n-6}$.

Așadar, cum fiecare al treilea termen din șirul lui Fibonacci este par, atunci $F_{n-3}$ și $F_{n-6}$ sunt ambii pari. Notăm $F_{n}$ cu $F_{p}(n)$, rezultă că $F_{n-3}$ este termenul par precedent, adică $F_{p}(n-1)$, iar $F_{n-6}$ este $F_{p}(n-2)$.
Înseamnă că $F_{p}(n)=4 \cdot F_{p}(n-1)+F_{p}(n-2), n \geq 2, F_{p}(0)=0$ și $F_{p}(1)=2$.

\section*{Varianta 31}

\section*{Indicații și răspunsuri}
1. Răspuns corect: d) 2024

Indicații: Expresia are valoare maximă dacă $\mathrm{n}=8080$ deoarece 8080 este cel mai mare număr natural de patru cifre multiplu de 2020.
2020 - n\%2020 + n/2020= 2020-0+4=2024
2. Răspuns corect: a) 234

Indicații: Nu căutăm decât cazurile când $y=2$. Pentru $x=0$ și $y=2$ se afișează 2 ; pentru $\mathrm{x}=1$ și $\mathrm{y}=2$ se afișează 3 ; pentru $\mathrm{x}=2$ și $\mathrm{y}=2$ se afișează 4 .
3. Răspuns corect: f) 42

Indicații: Se observă că toate elementele situate pe linia 2 sunt egale cu 1 și toate elementele situate pe linia 4 sunt egale cu 3. Suma elementelor situate pe linia 3 este 6 și suma elementelor situate pe linia 5 este 12 . În rest, elementele din tabloul bidimensional au valoarea 0 .
$1 * 6+3 * 6+2 * 3+4 * 3=42$
4. Răspuns corect: a) AUTONATICA

Indicații: Se elimină a doua literă O din BUTONOMATICA, apoi se înlocuiește litera B cu litera $\mathbf{A}$, apoi se elimină litera $\mathbf{M}$.
5. Răspuns corect: b) Limbajul C++/C (I.b<=J.a) || (J.b<=I.a)

Limbajul Pascal (I.b<=J.a) or (J.b<=I.a)
Indicații: Sunt două cazuri când $I$ și $\mathcal{J}$ nu se intersectează: fie când extremitatea dreaptă a lui I este mai mică sau egală cu extremitatea stângă a lui $\mathcal{J}$, fie când extremitatea dreaptă a lui $\boldsymbol{J}$ este mai mică sau egală cu extremitatea stângă a lui I.
6. Răspuns corect: b) 12

Indicații: Numerele căutate sunt pare deci au pe ultimele două poziții $00,02,10$, 12,20 , 22. Cifra sutelor poate să fie 1 sau 2 pentru că numerele sunt naturale și au exact 3 cifre. Deci sunt 2*6=12 numere.
7. Răspuns corect: c) 3080

Indicații: Pentru rapiditate se poate utiliza formula $n(n+1)(n+2) / 3$, unde $n=20$.
Deci $20 * 21 * 22 / 3=3080$.
8. Răspuns corect: b) $(16,9,7,5,4,3,2,1,0)$

Indicații: Primele 5 numere naturale pătrate perfecte: $0,1,4,9,16$. Sortate descrescător ajung: 16, $9,4,1,0$.
Primele 4 numere naturale prime: 2, 3, 5, 7 . Sortate descrescător ajung: 7, 5, 3, 2 .
După interclasare, C are 9 elemente și următorul conținut: ( $16,9,7,5,4,3,2,1,0$ ).
9. Răspuns corect: a) 192020

Indicații: Numai c își schimbă valoarea.
10. Răspuns corect: c) 514

Indicații: Toate nodurile de la 1 și până la 512 au câte doi descendenți direcți(fii). Nodul 513 are un singur fiu: 1026; toate nodurile de la 514 și până la 1026 sunt frunze.
1026-512=514
11. Răspuns corect: c) 9

Indicații: O soluție mai rapidă se bazează pe calculul invers: se scad cele șase subgrafuri care au mulțimea muchiilor nevidă din 15 (numărul total de subgrafuri ale grafului neorientat cu 4 noduri, $2^{4}-1=15$ ). Subgrafurile cu două muchii au mulțimea nodurilor
$\{1,2,3,4\}$, respectiv $\{1,2,3\}$ (două cazuri). Subgrafurile cu o muchie au mulțimea nodurilor $\{1,2,4\},\{1,2\},\{2,3,4\}$, respectiv $\{2,3\}$ (patru cazuri).
$2^{4}-1-2-4=9$
12. Răspuns corect: b) $O$ (logn), algoritm logaritmic

Indicații: Se utilizează un algoritm logaritmic bazat pe calculul sumei
$[n / 5]+[n /(5 * 5)]+[n /(5 * 5 * 5)]+\ldots$
13. Răspuns corect: e) 1

Indicații: Arcul pe care este suficient să îl adăugăm este (2,4).
14. Răspuns corect: c) $4^{9}$

Indicații: Numărul total de grafuri neorientate cu 8 noduri este $\mathbf{2}^{8 *(8-1) / \mathbf{2}}$
Numărăm ce nu variază deasupra diagonalei principale din matricea de adiacență asociată grafului deoarece graful este neorientat.
Numărăm perechile de noduri adiacente. Sunt 4 cazuri:
[2,8],[3,8],[5,8],[7,8]
Numărăm perechile de noduri neadiacente. Sunt 6 cazuri:
$[1,3],[1,5],[1,7],[3,5],[3,7],[5,7]$.
Deci numărul căutat este $2^{8 *(8-1) / 2-4-6}=2^{28-10}=2^{18}=4^{9}$
15. Răspuns corect: e) 8

Indicații: Tabloul unidimensional memorează răsturnatele primelor 33 de numere naturale pătrate perfecte. Deoarece căutăm câte numere din tablou se termină cu cifra $\mathbf{1}$, problema se rezumă la identificarea acelor pătrate perfecte care încep cu cifra 1 . Nu ne interesează numerele pătrate perfecte mai mari decât 200 și mai mici decât 999 pentru că nu încep cu cifra 1.
Pătratele perfecte căutate: $1,16,100,121,144,169,196,1024$, deci sunt 8 numere.

\section*{Varianta 32}
1. Răspuns corect: c)
2. Răspuns corect: a)

Indicații: Variabila i reţine, în ordine descrescătoare, multiplii comuni ai variabilelor a şi b. La final variabila i reţine cel mai mic multiplu comun al acesora.
3. Răspuns corect: a)

Indicații: Pentru x=9 se afişează: $1827 \quad 3645 \quad 54 \quad 63 \quad 728190$.
4. Răspuns corect: d)
5. Răspuns corect: c)

Indicații: se formează trei componente conexe, două cu câte trei noduri și una cu patru noduri.
6. Răspuns corect: b)

Indicații: Algoritmul lui Euclid reprezintă o metodă eficientă de calculare a celui mai mare divizor comun a două numere întregi.
7. Răspuns corect: d)

Indicații: Graful este conex și toate gradele sunt pare.
8. Răspuns corect: c)
9. Răspuns corect: b)

Indicații: Submulțimile generate sunt: $\{2\}\{1,2\}\{2,5\}\{2,9\}\{1,2,5\}$
$\{1,2,9\}\{2,5,9\}\{1,2,5,9\}$.
10. Răspuns corect: $c$ )
11. Răspuns corect: c)

Indicații: Pe nivelul 1 este un nod care are doi fii; pe nivelul 2 sunt 2 noduri care au, fiecare câte $\mathbf{3}$ fii; pe nivelul 3 sunt 6 noduri care au, fiecare câte $\mathbf{4}$ fii.
Deci, numărul de frunze este 24.
12. Răspuns corect: e)

Indicații: În antetul subprogramului f, $\mathbf{y}$ este parametru formal transmis prin referinţă.
13. Răspuns corect: $f$ )

Indicații: Se calculează cmmdc al numerelor $\mathbf{x}$ și $\mathbf{y}$ și apoi cmmdc pentru numărul rezultat și z.
14. Răspuns corect: a)
15. Răspuns corect: b)

Indicații: Ciclurile care sunt aceleași cu excepția punctului de plecare nu sunt luate în calcul separat.

\section*{Varianta 33}
1. Răspuns corect: a)
2. Răspuns corect: f)
3. Răspuns corect: d)
4. Răspuns corect: d)
5. Răspuns corect: c)

Indicații:
![](https://cdn.mathpix.com/cropped/2025_04_17_46e04c6acd873ea9558dg-313.jpg?height=193&width=277&top_left_y=592&top_left_x=542)
6. Răspuns corect: b)
\begin{tabular}{rlllllllll}
0 & 3 & 6 & 9 & 2 & 5 & 8 & 1 & 4 & 7 \\
2 & 5 & 8 & 1 & 4 & 7 & 0 & 3 & 6 & 9 \\
4 & 7 & 0 & 3 & 6 & 9 & 2 & 5 & 8 & 1 \\
6 & 9 & 2 & 5 & 8 & 1 & 4 & 7 & 0 & 3 \\
8 & 1 & 4 & 7 & 0 & 3 & 6 & 9 & 2 & 5 \\
0 & 3 & 6 & 9 & 2 & 5 & 8 & 1 & 4 & 7 \\
2 & 5 & 8 & 1 & 4 & 7 & 0 & 3 & 6 & 9 \\
4 & 7 & 0 & 3 & 6 & 9 & 2 & 5 & 8 & 1 \\
6 & 9 & 2 & 5 & 8 & 1 & 4 & 7 & 0 & 3 \\
Indicatii: & 8 & 1 & 4 & 7 & 0 & 3 & 6 & 9 & 2
\end{tabular}
7. Răspuns corect: b)

Indicații: $\frac{100(100-1)}{2}=4950$.
8. Răspuns corect: e)

Indicații: $\frac{b+c}{2}+\frac{c+d}{2}=\frac{b+d}{2}+c$.
9. Răspuns corect: c)

Indicații: Graful este conex și toate gradele sunt pare; nu există lanț hamiltonian.
10. Răspuns corect: c)

Indicații: Soluțiile sunt: $3+4+5$ și $3+9$.
11. Răspuns corect: a)

Indicații: Arborele are $\mathbf{n - 1}$ muchii și suma gradelor unui graf este dublul numărului de muchii.
12. Răspuns corect: b)
13. Răspuns corect: c)

Indicații: La fiecare pas se adaugă $3 \mathbf{k}$ și se scot $\mathbf{k}+\mathbf{2}$, adică ramân în coadă $\mathbf{2 k + 2}$ elemente. După 9 pași numărul de elemente este $2(1+2+\ldots+9)+2 * 9=90-18=72$.
14. Răspuns corect: a)
15. Răspuns corect: f)

Indicații: Arborele are $\mathbf{n - 1}$ muchii. Numărul de elemente nule din matricea de adiacenţă este: $n^{2}-2(n-1)=n^{2}-2 n+2$.
1. Răspuns corect: a)
2. Răspuns corect: c)
3. Răspuns corect: c )

Indicații: În timpul rulării, variabilei b are următoarele valori:
012478915222324.
4. Răspuns corect: a)
5. Răspuns corect: c)

Indicații:
![](https://cdn.mathpix.com/cropped/2025_04_17_46e04c6acd873ea9558dg-314.jpg?height=175&width=364&top_left_y=677&top_left_x=531)
6. Răspuns corect: a)
7. Răspuns corect: c)
8. Răspuns corect: d)
9. Răspuns corect: d)
10. Răspuns corect: c)
11. Răspuns corect: a)

Indicații: În antetul subprogramului $f, b$ este parametru formal transmis prin referinţă.
12. Răspuns corect: b)

Indicații: Graful complet cu n noduri are $\frac{\mathbf{n ( n - 1 )}}{\mathbf{2}}$ muchii.
13. Răspuns corect: c)
14. Răspuns corect: c )

Indicații: $\mathrm{S}_{1}=\mathrm{S}_{2}=\frac{n(n-1)}{2}$.
15. Răspuns corect: c )

\section*{Varianta 35}
1. Răspuns corect: b)
2. Răspuns corect: d)
3. Răspuns corect: b)

Indicații: Pentru fiecare $i \in\{1,2,3, \ldots, 10\}$, se realizează 10 execuţii ale instrucţiunii care afişează valoarea variabilei $\mathbf{k}$. Numărul final de execuţii este: 10•9=90.
4. Răspuns corect: c)
5. Răspuns corect: c)
6. Răspuns corect: d)
7. Răspuns corect: e)

Indicații: sunt necesare $11(11-1) / 2=55$ comparări şi $11(11-1) / 2=55$ interschimbări.
8. Răspuns corect: b)
9. Răspuns corect: f)
![](https://cdn.mathpix.com/cropped/2025_04_17_46e04c6acd873ea9558dg-315.jpg?height=50&width=1074&top_left_y=964&top_left_x=355)
10. Răspuns corect: c)
11. Răspuns corect: e)

Indicații: În antetul subprogramului $\mathbf{f}, \mathbf{y}$ este parametru formal transmis prin referință
12. Răspuns corect: b)

Indicații:
![](https://cdn.mathpix.com/cropped/2025_04_17_46e04c6acd873ea9558dg-315.jpg?height=174&width=297&top_left_y=1222&top_left_x=529)
13. Răspuns corect: a)
14. Răspuns corect: d)
15. Răspuns corect: b)

\section*{Varianta 36}
1. Răspuns corect: f) 2881 Indicații:
\begin{tabular}{|l|l|l|}
\hline \begin{tabular}{l} 
Operatori aritmetici binari \\
multiplicativi
\end{tabular} & Limbajul Pascal & Limbajul C|C++ \\
\hline Inmulțirea & * & $*$ \\
\hline Câtul împărțirii întregi & div & $/$ \\
\hline Restul împărțirii întregi & mod & $\%$ \\
\hline
\end{tabular}

Cei trei operatori au aceeași prioritate și se evaluează de la stânga la dreapta.
$\mathrm{x}=288$, $\mathrm{y}=1$.
2. Răspuns corect: $c$ ) ( $x=y$ ssi $y \neq z$ ) sau ( $x \neq y$ si $y=z$ )

Indicații: Cele trei variabile trebuie să fie inițializate. Expresia este $1 /$ True dacă cele trei variabile sunt egale sau diferite două câte două. Expresia este 0/False dacă oricare două variabile sunt egale și oricare două diferite.
3. Răspuns corect: e) 100

Indicații: $\mathbf{x}=\mathbf{y}=100$. Deoarece condiția $\mathbf{x > y}$ este falsă se execută $\mathbf{y} \leftarrow 10 * \mathbf{x}-8 * \mathbf{y}$. Valoarea lui $\mathbf{y}$ se modifică, $\mathbf{y}=\mathbf{2 0 0}$. Diferența absolută $|\mathbf{x}-\mathbf{y}|=|\mathbf{y}-\mathbf{x}|=100$.
4. Răspuns corect: $c$ ) $a>=1$

Indicații: Limbajul C++/C: Secvențele date sunt echivalente atunci când <condiția> din instrucțiunea while este la fel cu <condiția> din instucțiunea do... while.
Limbajul Pascal: Secvențele date sunt echivalente atunci când <condiția> din instrucțiunea while... do devine <negație condiție> în instrucțiunea repeat... until.
5. Răspuns corect: d) 96

Indicații: $f(5)=2 * f(4)=2 *(2 * f(3))=2 * 2 *(2 * f(2))=2 * 2 * 2 *(2 * f(1))$ $=2 * 2 * 2 * 2 *(2 * f(0))=2 * 2 * 2 * 2 *(2 * 3)=96$
6. Răspuns corect: d) strcat | concat

Indicații: Concatenarea a două șiruri se poate realiza în Limbajul C++/C cu subprogramul predefinit strcat, iar în Limbajul Pascal cu funcția predefinită concat.
7. Răspuns corect: a) 3

Indicații: Există 3 lanțuri distincte de lungime 3 de la nodul 1 la nodul 4:
$\mathrm{L} 1=[1,2,3,4]$, $\mathrm{L} 2=[1,2,5,4]$ și $\mathrm{L} 3=[1,5,2,4]$.
8. Răspuns corect: b) 1

Indicații: Primul nod este rădăcina arborelui. Fiecare nod are un singur descendent. Ultimul nod este frunză.
9. Răspuns corect: d) 207

Indicații: Numerele generate sunt: $108,126,153,162,18,207$ etc.
10. Răspuns corect: f) 2043231

Indicații: Elementele tabloului sunt: $a[0]=1, a[1]=2, a[2]=3 \ldots$ a [2020]=2021.
Suma elementelor este $s=\frac{n(n+1)}{2}$. Pentru $\mathbf{n}=2021$ suma elementelor este $\frac{2021 * 2022}{2}=$ 2043231.
11. Răspuns corect: e) 5
\begin{tabular}{|l|l|l|l|l|l|}
\hline 5 & 6 & 10 & 20 & 1 & tabloul inițial \\
\hline
\end{tabular}

Indicații: La fiecare parcurgere se compară elementele învecinate și se realizează interschimbul doar între elementele care nu respectă relația de ordine. Configurația este finală atunci când nu se mai realizează
\begin{tabular}{|l|l|l|l|l|l|}
\hline 5 & 6 & 10 & 1 & 20 & prima parcurgere \\
\hline 5 & 6 & 1 & $\Rightarrow 10$ & 20 & a 2-a parcurgere \\
\hline 5 & $1 \stackrel{1}{ }$ & 6 & 10 & 20 & a 3-a parcurgere \\
\hline 1 & 5 & 6 & 10 & 20 & a 4-a parcurgere \\
\hline 1 & 5 & 6 & 10 & 20 & a 5-a parcurgere \\
\hline
\end{tabular} niciun interschimb.
12. Răspuns corect: b) 29

Indicații: Numere excepționale: 110,111,112,113,114,115,116,117,118,19,210, 310, $410,510,610,710,810,910,221,331,242,441,551,661,771,283,881,392$, 991.
13. Răspuns corect: f) 3

Indicații: În schema apelurilor recursive, pentru $\mathrm{n}=3$, valorile afișate sunt încercuite: 111
21 3. Numerele asociate săgeților indică ordinea de executare a apelurilor recursive și, implicit, ordinea de afișare a valorilor.
![](https://cdn.mathpix.com/cropped/2025_04_17_46e04c6acd873ea9558dg-317.jpg?height=465&width=622&top_left_y=830&top_left_x=963)
14. Răspuns corect: c)

Indicații: Secvenţa interschimbă elementele triunghiurilor unu şi doi inclusiv elementele de pe diagonale, celelalte elemente păstrându-și poziția inițială.
\begin{tabular}{|l|l|l|}
\hline & Matricea inițială & Matricea finală \\
\hline ![](https://cdn.mathpix.com/cropped/2025_04_17_46e04c6acd873ea9558dg-317.jpg?height=249\&width=299\&top_left_y=1488\&top_left_x=359) & $$
\begin{array}{lllll}
1 & 2 & 3 & 4 & \frac{5}{1} \\
1 & \underline{2} & 3 & \underline{4} & \frac{5}{2} \\
1 & = & 3 & 4 & \frac{5}{5} \\
1 & 2 & 3 & 4 & 5 \\
1 & 2 & 3 & 4 & 5
\end{array}
$$ & 52341 54321 54321 54321 52341 \\
\hline
\end{tabular}
15. Răspuns corect: d) 252

Indicații: Pentru n număr natural format din 3 cifre, subprogramul va returna un număr natural format din aceleași 3 cifre. Dacă n conține cel puțin o cifră de 9 , atunci numărul returnat va avea cifra sutelor egală cu 9 .
În intervalul $[100,199]$ există 19 numere naturale care au cel puțin o cifră de 9 . În intervalul $[100,899]$ există $8 * 19=152$ numere naturale care au cel puțin o cifră de 9 . În intervalul $[900,999]$ există 100 de numere care au cel puțin o cifră de 9 .
Așadar, pentru $n \in[100,999]$ subprogramul poate returna 252 numere naturale cu cifra sutelor 9.

\section*{Varianta 37}
1. Răspuns corect: b)

Indicații: Se face diferenţa la nivel de cod ASCII (97- 99= -2).
2. Răspuns corect: c)

Indicații: Reprezentarea grafică a respectivului arbore este:
![](https://cdn.mathpix.com/cropped/2025_04_17_46e04c6acd873ea9558dg-318.jpg?height=318&width=240&top_left_y=600&top_left_x=395)
3. Răspuns corect: f)

Indicații: Matricea de adiacență este simetrică faţă de diagonala principală prin urmare numărul total de cifre 1 este $n(n-1)=n^{2}-n$. Cum pe diagonala principală avem un număr de $n$ cifre de 0 , iar numărul total de elemente din matrice este $n^{2}$, matricea desemnează încă de la bun început un graf norientat complet.
4. Răspuns corect: c)

Indicații: Pentru a obține un număr maxim de noduri izolate, vom asigura totalul de 10589 de muchii cu un număr minim de noduri (147). Cum restul nodurilor sunt izolate, gradul maxim pe care îl poate avea un nod din cele 147 este egal cu 146.
5. Răspuns corect: f)

Indicații: Pentru a ajunge la cuvântul din mijloc, se elimină primul cuvânt din şirul de caractere, iar apoi în cadrul structurii repetitive următoarele patru cuvinte. Ultimul cuvânt copiat în variabila c desemnează cuvântul căutat.
6. Răspuns corect: b)

Indicații: Variabila $\mathbf{k}$ va desemna numărul seriei de termeni în care se găseşte cel de pe poziţia $n$, iar $\mathbf{s}$ va reţine poziţia ultimului termen din acea serie.
7. Răspuns corect: c)

Indicații: Ex: pentru $n=5$, se va construi în memorie un tablou simetric faţă de ambele diagonale de forma:
23456
34565
45654
56543
65432
8. Răspuns corect: c)

Indicații: Se rețin în ordine descrescătoare multiplii comuni ai celor două variabile. La final variabila d va păstra valoarea celui mai mic multiplu comun al acestora.
9. Răspuns corect: d)

Indicații: Variabila c reţine puterea lui 5 din factorialul lui a, dar cum există în acelaşi timp şi un număr de elemente pare mai mare decat c, valoarea sa va desemna şi numărul de 0 obţinut din înmulţiri de forma 2*5.
10. Răspuns corect: $\mathbf{c}$ )

Indicații: Pentru a ajunge la o anumită literă din şirul de caractere strada, va trebui mai întâi accesat un element din tablou (v [5]), iar apoi câmpul adresa.
11. Răspuns corect: c)

Indicații: La prima deschidere a fişierului se vor citi toate valorile existente în acesta, iar la cea de-a doua se vor citi doar primele $\mathbf{n - 1}$ valori care includ şi valoarea citită initial în n .
12. Răspuns corect: a)

Indicații: Se caută valoarea raportului dintre suma cifrelor (obţinută prin apelul $f(n)$ ) şi numărul de cifre calculat în variabila c.
13. Răspuns corect: e)

Indicații: Se lipesc primele două caractere din şirul p la ceea ce a rămas în $\boldsymbol{s}$ după eliminarea de caractere.
14. Răspuns corect: b)

Indicații: Secvența parcurge în spirală în sensul acelor de ceasornic elementele tabloului respectiv.
15. Răspuns corect: e)

Indicații: Pentru a obține media dorită, variabila ev nu este validată dacă apelul funcţiei medie nu se regăseşte în intervalul de valori $[\mathbf{x}-0.5 ; \mathbf{x + 0 . 5}$ ).

\section*{Varianta 38}

\section*{1. Răspuns corect: f)}

Indicații: Termenii şirului lui Fibonacci: $1,1,2,3,5,8,13,21,34,55, \ldots$
Primii cinci termeni impari diferiţi duc la suma: $1+3+5+13+21=43$
2. Răspuns corect: b)

Indicații: Reprezentarea grafică a respectivului arbore este:
![](https://cdn.mathpix.com/cropped/2025_04_17_46e04c6acd873ea9558dg-320.jpg?height=350&width=451&top_left_y=638&top_left_x=379)
3. Răspuns corect: d)

Indicații: Numărul minim de încercări este obţinut la depistarea parolei încă de la prima testare, iar numărul maxim este dat de formula număr total caractere
$(52+10=62)^{\text {număr caractere parolă }}$
4. Răspuns corect: b)

Indicații: Muchia [1, 6] va deveni muchia [3,6] .
5. Răspuns corect: f)

Indicații: Calculul valorii variabilei d se va opri în momentul în care variabila i va ajunge la valoarea 0 .
6. Răspuns corect: f)

Indicații: Elementele de pe poziţii pare nu vor primi valori din fişier şi prin urmare elementul $\mathbf{v}$ [8] are valoarea 0 obținută din declararea vectorului ca parametru global.
7. Răspuns corect: d)

Indicații: Se vor afişa elementele tabloului pe 4 linii şi 3 coloane respectând formula de calcul $\mathbf{i + j}$.
8. Răspuns corect: e)

Indicații: Variabila $\mathbf{k}$ va trece prin toţi termenii de la 0 la 10, iar variabila $p$ va dezvolta pentru fiecare termen k înmulţirile cu $0,1, \ldots 10$.
9. Răspuns corect: a)

Indicații: Secvenţa va şterge pe rând fiecare apariţie a subşirului test în ordinea apariţiei acestora în şirul iniţial
10. Răspuns corect: c)

Indicații: Secvenţa foloseşte formula de calcul matematic al produsului dintre două matrice.
11. Răspuns corect: e)

Indicații: Variabila c va parcurge toate caracterele aflate între literele mici m şi r, dar la afişare se vor trece cele ce ocupa 5 poziţii în urmă, respectiv hijklm.
12. Răspuns corect: a)

Indicații: Secvenţa duce în prima parte la răsturnarea caracterelor din cadrul şirului de caractere, dar acestea vor fi repoziţionate în formatul iniţial în a doua jumătate a instruçiunii repetitive.
13. Răspuns corect: d)

Indicații: Se citesc pe rând datele corespunzătoare celor trei elevi, iar în paralel în s se calculează suma tuturor notelor din fişierul de intrare. Dacă din variabila s se scade valoarea 71, se va obține suma notelor lui Sebby, prin urmare expresia afişată reprezintă media acestuia.
14. Răspuns corect: a)

Indicații: Dacă diferenţa dintre componenta de pe poziţia curentă si cea anterioară nu respectă raţia dintre primele două componente, variabila ev va primi valoarea 0 ce indică o valoare invalidă.
15. Răspuns corect: c)

Indicații: Se parcurg în paralel cele două diagonale şi se interschimbă fiecare element de pe diagonala principală( $\mathbf{a}_{\mathbf{i}}, \mathbf{i}$ ) cu fiecare element de pe cea secundară ( $a_{i, n-i+1}$ ).

\section*{Indicații și răspunsuri}
1. Răspuns corect: c) a și -1

Indicații: Din numărul total de valori din șir, 100, elimină numărul cifrelor din șir
2. Răspuns corect: e) VBPRE
3. Răspuns corect: e) patru

Indicații: Primele patru elemente din tablou primesc valoarea 9
4. Răspuns corect: c) 2

Indicații: după prima parcurgere 51 ajunge pe poziția finală, după cea de-a doua parcurgere 40 ajunge pe poziția finală
5. Răspuns corect:

Limbajul C++/C b) (i<j) \& \& (i+j<n+1)
Limbajul Pascal b) (i<j)AND (i+j<n+1)
Indicații: Condiția stabilește o intersecție pe cele două zone: zona aflată deasupra diagonalei principale ( $\mathrm{i}<\mathrm{j}$ ) și zona aflată deasupra diagonalei secundare ( $\mathrm{i}+\mathrm{j}<\mathrm{n}+1$ )
6. Răspuns corect: e) 12

Indicații: Graful neorientat cu 8 noduri şi 28 de muchii este un graf complet. Pentru un număr minim de muchii eliminate se aleg 2 componente conexe astfel: o component conexă cu 2 noduri și o componentă conexă cu 6 noduri, se vor elimina $6+6=12$ muchii.
7. Răspuns corect:

Limbajul C++/C a) $x^{*} y>y^{*} z \quad \& \& \quad x^{*} z>y^{*} z$
Limbajul Pascal a) ( $x^{\star} y>y^{\star} z$ ) AND ( $x^{*} z>y^{\star} z$ )
8. Răspuns corect:

Limbajul C++/C b) ( $x>1000$ ) \&\& ( $\left.\left(x^{*} x^{*} x\right) \% 1000==0\right)$
Limbajul Pascal b) ( $x>1000$ ) AND ( $\left(x^{*} x^{*} x\right)$ MOD $\left.1000=0\right)$
Indicații: $x=36 * 35=1260$
Răspuns corect: f) [ $\left.\log _{2} \mathrm{n}\right]+1$
Răspuns corect: d) 45
Răspuns corect: c) este conex și suma elementelor de pe fiecare coloană a matricei de adiacență este număr par
Răspuns corect: a) $8 \quad 7 \quad 20 \quad 12$
Răspuns corect: e) 12600
Indicații: $C_{10}^{1} * C_{9}^{2} * C_{7}^{3 *} C_{4}^{4}=10 * 36 * 35 * 1=12600$ șiruri distincte
14. Răspuns corect: b) 8

Indicații: Pentru fiecare nod ales drept nod rădăcină, există un singur vector de tați
Răspuns corect: b) verifică dacă numărul $\mathbf{x}$ este divizibil cu b-1
15.

Indicații: Se aplică criteriul de divizibilitate: un număr natural scris în bază b se divide cu b-1 dacă și numai dacă suma cifrelor sale este un multiplu de b-1.

\section*{Varianta 40}

\section*{Indicații și răspunsuri}
1. Răspuns corect: d) 63

Indicații: $f(63)=f(62)+63=\ldots=f(4)+5+6+\ldots+63=8+(5+6+\ldots+63)=2014$
2. Răspuns corect: e) 50
3. Răspuns corect: c) studentarterou

Indicații: La primul șir se concatenează cel de-al doilea șir, mai puțin primul caracter, apoi se concatenează cel de-al treilea șir, mai puțin primele două caractere.
4. Răspuns corect: d) oricare ar fi $\mathrm{x}, \mathrm{y}, \mathrm{z}, \mathrm{p}$ egal cu q

Indicații: cele două expresii sunt echivalente
5. Răspuns corect: c) 130

Răspuns corect: a) $\left[-2^{\mathrm{n}-1}, 2^{\mathrm{n}-1}-1\right]$
Indicații: primul bit, din reprezentare, este cel de semn (0- pentru numere întregi pozitive șii 1- pentru numere întregi negative), ceilalți $n-1$ biți sunt folosiți pentru reprezentarea valorii absolute a numărului.
7. Răspuns corect: b) 16

Indicații: se intră o singură dată în instrucțiunea while, variabila $\mathbf{p}$ nu se modifică
8. Răspuns corect: b) $7,16,10$

Indicații: se înjumătățește secvența curentă în care se face căutarea
9. Răspuns corect: e) 777

Indicații: Numărul valorilor de 1 din tabloul bidimensional, pe linii, este $1+2+4+8+$ $16+32+64+128=255$. Tabloul are 8 linii și 129 coloane. De unde, numărul de valori 0 este: $(129-1)+(129-2)+\ldots+(129-128)=777$
10. Răspuns corect: a) 377

Indicații: se pot folosi termenii din șirul lui Fibonacci
11. Răspuns corect: c) 11

Indicații: Graful neorientat are 20 de muchii care formează o componentă conexă folosind 7 noduri. Rămân 10 noduri izolate.
12. Răspuns corect: b) 13

Indicațiii: 2 și toate numerele impare cuprinse între 3 și [sqrt(681)]
13. Răspuns corect: d) 101

Indicații: Numărul total de permutări cu 5 elemente este $5!=120$. După permutarea 51423 se mai generează încă 19 termeni.
14. Răspuns corect: b) $A, B$

Indicații: c.m.m.m.c(m,n)=m*n/c.m.m.d.c(m,n)
15. Răspuns corect: f) $4^{13}$

Indicații: Numărul grafurilor neorientate cu 8 noduri este $2^{28}$. Numărul grafurilor neorientate cu 8 noduri, în care nodurile 2 și 3 sunt neadiacente, este $2^{28} / 2=2^{27}$, s.a.m.d.

\section*{Varianta 41}

\section*{Indicații și răspunsuri}
1. Răspuns corect:a) exact $n(n-1) / 2-m$

Indicații: Graful G1 va conţine muchiile grafului complementar al lui G (dacă în G există o muchie în G1 nu va exista). Din graful complet cu n noduri şi n(n-1)/2 muchhi se scad muchiile grafului $G$, $m$
2. Răspuns corect: f) $\mathrm{x}<=\mathrm{d}$

Indicații: Funcţia descompune în factori primi un număr. Când numărul $x$ devinde egal cu divizorul atunci numărul $x$ este un factor prim la puterea 1
3. Răspuns corect: d) 4

Indicații: Se parcurge vectorul pentru căutarea valorii x, repetiția oprindu-se la prima apariţie a valorii $x$ în vector sau după ce toate elementele au fost parcurse dacă x nu apare în vector.
4. Răspuns corect: d) 128

Indicații: Există şirul de apeluri: $\mathrm{F}(7), \mathrm{F}(6) \ldots \mathrm{F}(0)$ care duce la valorile returnate de la stânga la dreapta $1,2 * 1,2 * 2,2 * 4,2 * 8,2 * 16,2 * 32,2 * 64=128$
5. Răspuns corect: c) CDEFGEFG

Indicații. Se memorează şirul de caractere începând de la poziţia 4 (C) 5 (Pascal) apoi şirul începând de la poziţia 2 (C) | 3 (Pascal) şi apoi se concatenează aceste două şiruri.
6. Răspuns corect: d) B,C

Indicații: La evaluare se ține cont de prioritatea operatorilor
7. Răspuns corect: b) BEC BED CAB

Indicații: Se pleacă de la variantele propuse şi aplicând metoda backtracking se generează următoarele soluţii.
8. Răspuns corect: a) de 5 ori
9. Răspuns corect: f) 0

Indicații: Fiecare nod $n$ are ca fii nodurile $2 n$ şi $2 n+1$. Fiind număr impar nu rămâne nici un nod cu un fiu.
10. Răspuns corect: $e) n(n-1) / 2$

Indicații: Inițial vectorul este ordonat crescător, deci se face numărul maxim de interschimbări.
11. Răspuns corect: $\mathbf{c}$ ) $\mathrm{n}=5, \mathrm{U}=\{[1,3],[1,4],[3,4],[2,4],[4,5],[2,5]\}$

Indicații: Se desenează fiecare graf. Ca să fie eulerian trebuie ca să existe un ciclu care să conțină toate muchiile grafului o singură dată iar ca să nu fie hamiltonian trebuie să nu existe niciun ciclu care să conţină toate nodurile grafului o singură dată.
12. Răspuns corect:

Limbajul C++/C b) if( $x>y \& \& y>z$ ) $p=x^{*} y^{*} z ;$

Limbajul Pascal b) if ( $x>y$ ) AND ( $y>z$ ) then $\mathrm{p}:=\mathrm{x}^{*} \mathrm{y}^{*} \mathrm{z}$;
Indicații: Din proprietatea de tranzitivitate se observă că expresia $\operatorname{logică~} \mathrm{z}>\mathrm{x}$ are întotdeauna valoare 1 |true.
13. Răspuns corect: b) cuprins între 7 și 12

Indicații: Numărul de înjumătăţiri este $\log 1000$
14. Răspuns corect: a) A

Indicații: Se foloseşte algoritmul care modifică numărul prin adunarea ultimei cifre la câtul împărțirii la 10 până când se obține un număr care are o singură cifră.
15. Răspuns corect: c) 94

Indicații: $1+3+3 x 2+3 x 2^{2}+3 x 2^{3}+3 x 2^{4}=94$

\section*{Varianta 42}

\section*{Indicații și răspunsuri}
1. Răspuns corect: d) 6

Indicații: Subprogramul se apelează pe prima jumătate (de la n la $(\mathrm{n}+\mathrm{m}) / 2$ ) şi pe a doua $((\mathrm{n}+\mathrm{m}) / 2+1$ până la m$)$ adunând restul fiecărui element la imparitrea cu 2.
2. Răspuns corect: a) Numai S1

Indicații: S1 funcţionează cât timp există litera în şir şi cât timp caracterele sunt diferite de litere mari. Asemeni cel de-al doilea. Când se opresc afișează caracterul la care s-au oprit.
3. Răspuns corect: d) $1+2=3$

Indicații: Se trece peste primele două litere , când se ajunge la primul 1 se schimbă următorul cu caracterul + . Se trece la primul caracter de 2 , următorul după acesta fiind schimbat în caracterul =, urmând ca mai apoi să rămână la final doar caracterul 3.
4. Răspuns corect: d) trei

Indicații: Se parcurg elementele până la întâlnirea lui 0 , pentru $\mathrm{i}>0$.
5. Răspuns corect: f) Cea mai lungă secvență de valori de parități diferite

Indicații: Se parcurge vectorul numărând elementele consecutive de pafități diferite. Dacă se ajunge la un element de aceeași paritate cu cel de dinainte se resetează lungimea secvenței și se analizează lungimea maximă.
6. Răspuns corect: f) 62
7. Răspuns corect: e) $C_{n-p+1}^{2}$
8. Răspuns corect: c) $\mathrm{j}^{*} \mathrm{j}+$ suma( $\mathrm{j}^{*} \mathrm{j}-1$ )
9. Răspuns corect: b) 1

Indicații: Se repetă o singură dată pentru că se evaluează v[0], care este 0 , deci repetițiea se oprește, și apoi se incrementează i.
10. Răspuns corect: d) $2^{5}$

Indicații: Pe nivelul 0, se găsește un nod (rădăcina), pe nivelul 1, 2 noduri, pe nivelul 2, 4 noduri, pe nivelul 3, 8 noduri, pe nivelul 4, 16 noduri iar nivelul 532 noduri.
11. Răspuns corect: d) 1023; 1032; 105; 1203;

Indicații: Se pornește de la o soluție propusă și se generează cu algoritmul backtracking celelalte soluții.
12. Răspuns corect: a) metoda căutării binare

Indicații: Sunt 2 metode de căutare: secvențială și binară. Eficientă este metoda căutării binare pentru care se efectuează $\log (n)$ operații.
13. Răspuns corect: d) permutărilor

Indicații: Se folosește metoda backtracking. Coloanele din matrice vor reprezenta elementele care se generează.
14. Răspuns corect: f) 20

Indicații: Vor exista doar muchii în care extremitatea inițială este mai mică decât extremitatea finală. Deci, dacă există drumul de la i la j, în care nodurile sunt în ordine crescătoare, nu va exista drumul de la j la i. Deci, fiecare nod formează o componentă tare conexă.
15. Răspuns corect: c) 13

Indicații: Se numără numărul de înjumătățiri care se fac pentru a ajunge la 1.

\section*{Varianta 43}

\section*{Indicații și răspunsuri}
1. Răspuns corect: b) 1

Indicații: Se execută operațiile matematice în ordinea priorității operatorilor.
2. Răspuns corect: e) 3.15

Indicații: Variabilelor i și j, fiind de tip întreg, li se atribuie [x] și, respectiv, [y].
3. Răspuns corect: d) 0

Indicații: Se evaluează $\mathbf{j} \neq 0(\mathrm{~F})$ și apoi operatorul de incrementare $=>\mathrm{i}=1$. Pentru că $\mathrm{i} \neq 0$ se decrementează $=>\mathrm{i}=0$ și apoi se evaluează suma.
4. Răspuns corect: e) 51970

Indicații: Se introduc în x cifrele impare din n , în aceeași ordine, doar că se începe de la cifra zecilor.
5. Răspuns corect: f) 8

Indicații: Se calculează ultima cifră nenulă a numărului n!.
6. Răspuns corect: d) 2856413618269025

Indicații: Se șterg elementele nule din tabloul unidimensional.
7. Răspuns corect: b) Informatica-poli

Indicații: Se determină adresa de memorie a caracterului '-', împărțindu-se astfel șirul inițial în două șiruri de caractere. Apoi se concatenează al doilea șir cu primul, după ce prima literă a fiecărui șir se transformă.
8. Răspuns corect: b) b

Indicații: Se declară un tablou unidimensional cu 2 elemente de tip structură, fiecare element conținând un pointer către un șir de caractere. Câmpului din a doua structură i se atribuie șirul definit de la poziția 1 .
9. Răspuns corect: c) A și D

Indicații: Pentru a afla numărul de drumuri de lungime k dintre două noduri i și j într-un graf orientat, se calculează matricea $X=A^{k}$, unde $A$ este matricea de adiacență. $X_{i, j}$ reprezintă numărul de drumuri. Deci $\mathrm{A}=\left(\begin{array}{llll}0 & 1 & 1 & 0 \\ 0 & 0 & 1 & 1 \\ 0 & 0 & 0 & 1 \\ 1 & 1 & 0 & 0\end{array}\right)$ și $\mathrm{A}^{3}=\left(\begin{array}{llll}2 & 2 & 0 & 1 \\ 1 & 1 & 0 & 1 \\ 0 & 1 & 2 & 1 \\ 1 & 1 & 1 & 3\end{array}\right)$ Se observă că între $\mathrm{A}^{3}[1,3]=\mathrm{A}^{3}[3][1]=0$.
10. Răspuns corect: d) 4

Indicații: Șirul apelurilor este: $f(0) \stackrel{0<7}{\Rightarrow} f(2) \xlongequal{2<7} f(4) \stackrel{4<7}{\Longrightarrow} f(6) \xrightarrow{6<7} f(8) \stackrel{8>7}{\Longrightarrow} 8-3$ raspuns care se returnează la apelul anterior, adunându-se de fiecare dată $1 . f(0)=7$. Apoi se calculează $f(7)=2 ; f(2)=6$ și $f(6)=4$.
11. Răspuns corect: c) 24153
$$
\begin{aligned}
& 51423 \\
& 12345 \\
& 54321 \\
& 41532
\end{aligned}
$$

Indicații: Se fac 3 permutări circulare ale ultimelor 3 linii astfel încât matricea la final revine la configurația inițială.
12.

Răspuns corect: c) 7196 7198

Indicații: Dacă se pornește de la 7196 se continuă cu 7197 (incorectă, se repetă cifra 7), 7198 , se revine la a doua cifrăși se alege 2 , dar fiind prim nu e permis, la fel 3 și se ajunge la 4 , apoi se continuă cu 0 și cu 1 , adică 7401 .
13. Răspuns corect: c) II și IV

Indicații: Însecvența IV, există un nod cu gradul 8, ceeace nu este posibil într-un graf cu 8 noduri. Însecvența II, gradul 6 al celor 4 vârfuri înseamnă 18 muchii (nodul 1 incident cu 6 muchii, nodul 2 incident cu 5 muchii (a șasea fiind (1,2), deja numărată)) etc. Adică
$6+5+4+3=18$ muchii. Un exemplu posibil de graf ar fi:
![](https://cdn.mathpix.com/cropped/2025_04_17_46e04c6acd873ea9558dg-329.jpg?height=267&width=348&top_left_y=1325&top_left_x=1395) Din secvența dată: $6+6+6+6+3+3+2+2=34=>17$ (muchii contradicție).
14. Răspuns corect: b) $\mathrm{O}(\mathrm{n})$ pentru f 1 și $\mathrm{O}\left(2^{\mathrm{n}}\right)$ pentru f2

Indicații: Am puteascriepentrufiecarefuncție, complexitatea timp, recurent, astfel:
$\mathrm{T}_{1}(\mathrm{n})=\mathrm{T}_{1}(\mathrm{n}-1)+\mathrm{C}$ care este $\mathrm{O}(\mathrm{n})$
$\mathrm{T}_{2}(\mathrm{n})=2 \cdot \mathrm{~T}_{2}(\mathrm{n}-1)+\mathrm{C}$ care este $\mathrm{O}\left(2^{\mathrm{n}}\right)$
15. Răspuns corect: a) $\mathrm{O}\left(\log _{2} k\right)$

Indicații: Se calculează $\mathrm{n}^{\mathrm{k}}$ în timp logaritmic după următoarea metodă:
$$
n^{k}=\left\{\begin{array}{c}
n \cdot\left(n^{2}\right)^{\frac{k-1}{2}}, \text { daca } k \text { impar } \\
\left(n^{2}\right)^{\frac{k}{2}}, \text { daca } k \text { par }
\end{array}\right.
$$

\section*{Varianta 44}

\section*{Indicații și răspunsuri}
1. Răspuns corect: c) 10

Indicații: Se fac operațiile matematice în ordinea priorității operatorilor.
2. Răspuns corect: e) 3.5

Indicații: Se folosește operatorul cast, astfel încât rezultatul împărțirii i/j să fie real.
3. Răspuns corect: f) 5

Indicații: Se evaluează $\mathrm{j} \neq 0$ (A)și apoi operatorul de decrementare $=>\mathrm{j}=2$. Pentru că $!\mathrm{j}=0$ atunci se evaluează $\mathrm{i} \neq 0(\mathrm{~A})$ și se incrementează $\mathrm{j}=>\mathrm{j}=3$.
4. Răspuns corect: d) 3

Indicații: Se concatenează la sfârșitul șirului s șirul ABCDE și se obține ABCDEABCDE, apoi se șterg primele 3 caractere. Deci șirul s este DEABCDE.
5. Răspuns corect: b) 1

Indicații: Apelul funcției implică decrementarea valorii parametrului și returnarea acestui rezultat.
6. Răspuns corect: a) $p^{*}=2 \mid p:=p * 2$

Indicații: Fie $\mathrm{n}=d_{1}{ }^{e_{1}} \cdot d_{2}{ }^{e_{2}} \cdot \ldots \cdot d_{r}{ }^{e_{r}}$, descompus în factori primi. Ș̦tim că numărul de divizori ai lui $n$ este $\left(e_{1}+1\right) \cdot\left(e_{2}+1\right) \cdot \ldots \cdot\left(e_{r}+1\right)$. Folosindu-se acest rezultat, se determină factorii primi până la $\sqrt{n}$. Dacă $n \neq 1$ atunci $n$ este un număr prim, deci la produsul anterior mai trebuie înmulțit 2.
7. Răspuns corect: e) $x[k / 2]^{*}(x[k / 2]-1) / 2 \mid x[k \operatorname{div} 2]^{*}(x[k \operatorname{div} 2]-1) \operatorname{div} 2$

Indicații: Se folosește vectorul de frecvență $x$ astfel încât, x[i] să exprime numărul de elemente din v care au restul i la împărțirea cu k. Se observă că algoritmul nu adaugă numărul de perechi obținute între elemente cu același rest [ $\mathrm{k}: 2$ ], în cazul în care $k$ este par. Acest număr reprezintă numărul de combinări ale valorilor de rest [ $\mathrm{k}: 2]$.
8. Răspuns corect: $c) a[n-j-1][n-i-1]=2 ; \quad a[n-j-1, n-i-1]:=2$;

Indicații: Se parcurg elementele din cadranul 1 și apoi prin simetrie față de diagonala secundară se vor completa și elementele din cadranul II.
9. Răspuns corect: c) -1

Indicații: Șirul apelurilor este: $\mathrm{f}(16) \stackrel{16>8}{\Longrightarrow} \mathrm{f}(13) \stackrel{13>8}{\Longrightarrow} f(10) \stackrel{10>8}{\Longrightarrow} \mathrm{f}(7) \stackrel{7<8}{\Longrightarrow} 7-5=2$. Se calculează $f(2)=-3$, apoi se revine cu $-3+4$, se calculează $f(1)=-4$ ș.a.m.d.
10. Răspuns corect: c) 8

Indicații: Se observă că între nodurile 1, 2, 6 există drum între oricare două noduri, deci ele formează o componentă tare conexă. Între celelalte 7 noduri nu există drum de la i la j și de la j la i, deci fiecare nod formează separat, câte o componentă tare conexă.
Răspuns corect: e) apnmdc
Indicații: Plecând de la soluția apnmdc se generează următoarele încercări: apnmdd, apnmdm, apnmdn, apnmdp... și se revine la primul caracter și se generează ebacid

Răspuns corect: b) se elimină o muchie și se adaugă două
Indicații: Se observă că graful are două componente conexe. Fiecare componentă conexă are 501 noduri, fiind subgrafuri complete. Deși gradul fiecărui nod este par, graful nu este eulerian pentru că nu este conex. Dacă, de exemplu, se elimină muchia [2,4] din subgraful cu noduri pare, atunci gradele celor două noduri devin impare, deci trebuie refăcută paritatea, așa că putem adăuga, de exemplu, muchia [2,1], dar și [4,1]. În felul acesta, graful devine conex și se păstrează paritatea gradelor nodurilor.
13. Răspuns corect: d) 7

Indicații: Pentru ca înălțimea arborelui să fie minimă, numărul de fii ai fiecărui nod trebuie să fie maxim, adică 4 . Deci pe nivelul 0 se găsește rădăcina, pe nivelul 1 vor fi $4^{1}$ noduri, pe nivelul $2,4^{2}=16$ noduri, pe nivelul $3,4^{3}=64$ noduri, apoi $4^{4}=256,4^{5}=1024$, pe nivelul 6, restul nodurilor. Deci, în total, 7 nivele.
14. Răspuns corect: d) $O$ (nlogn)

Indicații: Pentru prima repetiție timpul este $\mathrm{O}(\mathrm{n})$, dar pentru a doua, timpul este logaritmic.
15. Răspuns corect: b) n

Indicații: Expresia se poatere scrie astfel:
$a_{0}+a_{1} * x+a_{2} * x^{2}+a_{3} * x^{3}+\ldots+a_{n} * x^{n}=a_{0}+x^{*}\left(a_{1}+x^{*}\left(a_{2}+\ldots . .+x^{*}\left(a_{n-1}+x^{*} a_{n}\right)\right) ..\right)$

\section*{Varianta 45}

\section*{Indicații și răspunsuri}
1. Răspuns corect: d) $\mathbf{1 5}$

Indicații: Întâi se evaluează operatorul de decrementare, deci valoarea variabilei i devine 3, se face produsul i*j și apoi se evaluează operatorul de incrementare.
2. Răspuns corect: c) -4

Indicații: Întâi se face produsul, deci $\mathbf{k}=-9$, apoi la $\mathbf{k}$ se adună $\mathbf{j}=>\mathbf{k}=\mathbf{- 1 2}$ și se determină câtul împărtirii lui $\mathbf{k}$ la $\mathbf{i}$.
3. Răspuns corect: b) -1

Indicații: Se evaluează $\mathbf{j} \neq 0$ și apoi operatorul de decrementare $=>\mathbf{i}=1$. Pentru că $\mathbf{i} \neq 0$ se mărește $\mathbf{j}$ cu 1 => $j=-1$ și apoi se evaluează produsul.
4. Răspuns corect: d) 21

Indicații: Se returnează valoarea inițială a parametrului și apoi se incrementează parametrul, pentru că operatorul de incrementare este în formă postfixată. Deci se returnează 1 iar parametrul va avea valoarea 2 .
5. Răspuns corect: e) 4

Indicații: Este declarat un vector cu 2 elemente de tip structură, fiecare element conținând un alt vector cu două numere întregi. Se atribuie valoare doar elementelor S[0].a[1] și S[1].a[0].
6. Răspuns corect: f) $\mathbf{3 0}$

Indicații: Algoritmul determină toate numerele din intervalul [1, 10000] care sunt formate doar din cifrele 4 și/sau 6. Exemple 4, 6, 44, 46, 64, 66 etc.
7. Răspuns corect: c) II

Indicații: Se parcurg elementele din zona I, dar se folosesc elementele simetrice cu acestea față de diagonala secundară, deci elemente din zona II
8. Răspuns corect: b) Automatica-UPB

Indicații: Se determină adresa de memorie poziția caracterului '-', împărțindu-se astfel șirul inițial în două șiruri de caractere. Apoi se concatenează al doilea șir cu primul, după ce prima literă a celui de al doilea șir se transformă în literă mare.
9. Răspuns corect: f) $a[i]>a[j]$

Indicații: Se folosește algoritmul de ordonare prin numărare. Fiecare element b[i] memorează numărul de elemente din tabloul unidimensional a, mai mari decât a[i].
10. Răspuns corect: a) - 6

Indicații: Șirul apelurilor este: $f(19,7) \stackrel{19>7}{\Longrightarrow} f(16,8) \xrightarrow{16>8} f(13,9) \xrightarrow{13>9} f(10,10) \xrightarrow{10=10} f(11$, $10) \stackrel{11>10}{ } f(8,11) \stackrel{8<11}{\Longrightarrow} 3 * 8-2^{*} 11$ raspuns care se returnează la apelul anterior șamd, scăzându-se 2 , unde e cazul.
11. Răspuns corect: c) $4316 \quad 3618 \quad 3418$

Indicații: O abordare ar putea să plece de la o variantă dată. Folosind metoda backtracking, ținând cont de condițiile impuse de problemă, se pleacă de la primul număr propus și se generează următoarele trei soluții.
12. Răspuns corect: a) $\mathbf{2}^{\text {i-1 }} \mathbf{- 1}$

Indicații: Pentru ca pe niveluri, să avem număr maxim de noduri, trebuie ca toate nodurile de pe nivelurile anterioare sa aibă câte 2 fii. Deci, pe nivelul 1, avem $\mathbf{2}^{\circ}$ noduri, pe nivelul 2 sunt $\mathbf{2}^{\mathbf{1}}$ noduri, pe nivelul $\mathbf{3}$ sunt $\mathbf{2}^{\mathbf{2}}$ noduri ș.a.m.d. Deci, pe nivelul i vor fi $2^{\text {i-1 }}$ noduri. Cum numărul total de noduri este par, iar pe nivelul 1 este un nod, atunci pe nivelul i vor fi $\mathbf{2}^{\mathbf{i}-1} \mathbf{- 1}$ noduri.
13. Răspuns corect: c) $\mathbf{2 5}$

Indicații: Partițiile mulțimii nodurilor pot avea:
prima mulțime - 1 nod; a doua - 9 noduri $=>$ nr de muchii $1 * 9$
prima mulțime - $\mathbf{2}$ noduri; a doua - $\mathbf{8}$ noduri $=>$ nr de muchii 2 * 8
prima mulțime - 5 noduri; a doua - 5 noduri => nr de muchii 5*5 (maxim)
14. Răspuns corect: b) $O$ ( $n$ )

Indicații: Complexitatea algoritmului pare a fi $\mathrm{O}\left(\mathrm{n}^{2}\right)$. Pentru că a doua repetiție nu resetează valoarea lui j la 0 , atunci vor fi valori ale lui i pentru care a doua repetiție nu se mai execută . Deci, vom avea maxim 2 treceri prin vector, deci O(n).
15. Răspuns corect: e) $\mathbf{3 0}$

Indicații: Pentru a afla numărul de drumuri de lungime $\mathbf{k}$ dintre două noduri $\mathbf{i}$ și $\mathbf{j}$ întrun graf orientat, se calculează matricea $\mathbf{X}=\mathbf{A}^{\mathbf{k}}$, unde $\mathbf{A}$ este matricea de adiacență. $\mathbf{X}_{\mathbf{i}, \boldsymbol{j}}$ reprezintă numărul de drumuri dintre nodurile $\mathbf{i}$ și $\mathbf{j}$ de lungime $\mathbf{k}$. Pentru a calcula numărul total de drumuri din digraf, atunci se adună toate valorile matricei $\mathbf{X}$. Deci
$\mathrm{A}=\left(\begin{array}{llll}0 & 1 & 1 & 0 \\ 1 & 0 & 1 & 1 \\ 1 & 0 & 0 & 1 \\ 1 & 0 & 0 & 0\end{array}\right), . ., \mathrm{A}^{3}=\left(\begin{array}{llll}3 & 2 & 2 & 1 \\ 3 & 2 & 3 & 2 \\ 2 & 1 & 2 & 2 \\ 2 & 0 & 1 & 2\end{array}\right)$. Se adună toate elementele și se obține 30.

\section*{Varianta 46}

\section*{Indicații și răspunsuri}
1. Răspuns corect: c) 6

Indicații: Se fac operațiile în ordinea priorităț̦ii operatorilor
2. Răspuns corect: e) $\mathbf{3 2}$

Indicații: Sunt generate toate numerele care conțin doar cifrele 5 și/sau 7.
3. Răspuns corect: d) $4852 \quad 26114165$

Indicații: Se rearanjează elementele vectorului, astfel încât cele pare să fie la începutul șirului, iar cele impare la sfârșit.
4. Răspuns corect: d) informaticatest

Indicații: Funcția strtok (C) separă șirul inițial în două șiruri. Se concatenează al doilea șir cu primul și se modifică șirul inițial. Nu se mai adaugă spațiu între cele două șiruri.
5. Răspuns corect: $f$ ) $r>t$

Indicații: Pentru intersecția a $\mathbf{n}$ intervale se determină maximul dintre capetele din stânga ale intervalelor (notat cu $\boldsymbol{r}$ în secvența dată) și minimul dintre capetele din dreapta ale intervalelor (notat cu t în secvența dată). Intervalele se intersectează dacă $r \leq t$.
6. Răspuns corect: f) $a[n+1-j][n+1-i] \quad \mid a[n+1-j, n+1-i]$

Indicații: Algoritmul parcurge elementele din zona I și, prin simetrie față de diagonala secundară, accesează elementele din zona II.
7. Răspuns corect: a$) \mathrm{p} /=2 \mid \mathrm{p}$ div 2

Indicații: Este folosit algoritmul căutării binare.
8. Răspuns corect: c) 3

Indicații: Variabila $\mathbf{S}$ de tip înregistrare, după toate atribuirile făcute, va avea valoarea $\{4,5,6,\{3,2,1\}\}$. Se afișează rezultatul expresiei: 5-2
9. Răspuns corect: a) 0

Indicații: Șirul apelurilor este: $\mathrm{f}(6,2) \stackrel{6>2}{\Longrightarrow} \mathrm{f}(2,6) \stackrel{2<6}{\Longrightarrow} 4$, apoi se revine la apelul anterior și se evaluează $f(4,3) \stackrel{4>3}{\Longrightarrow} f(3,4) \stackrel{3<4}{\Longrightarrow} 1$; se continuă cu $f(1,1) \stackrel{1=1}{\Longrightarrow} f(2,1) \stackrel{2>1}{\Longrightarrow} f(1,2) \stackrel{1<2}{\Rightarrow} 1$ și se revine adunând pe 2 sau scăzând pe 1 .
10. Răspuns corect: e) 8

Indicații: Circuitele elementare sunt: ABA, ACBA, ACDBA, ADBA, ADCBA, AEA, AEDBA, AEDCBA

\section*{11. Răspuns corect: b) egrl egrm}

Indicații: O abordare ar putea să plece de la o variantă dată. Folosind metoda backtracking, ținând cont de condițiile impuse de problemă, se pleacă de la primul cuvânt propus și se generează următoarele două soluții.
12. Răspuns corect: c) $\mathbf{x}=\mathbf{t}[\mathbf{x}]$; | $\mathbf{x}:=\mathrm{t}[\mathbf{x}]$;

Indicații: Se merge pe drumul de la nodul $\mathbf{x}$ către rădăcină, adică, de fiecare dată, se trece de la $\mathbf{x}$ la tatăl lui.
13. Răspuns corect: a) 1100

Indicații: Algoritmul determină cifra de control a numărului $n$. Deci, se cere să se determine câte numere din intervalul [100, 10000] au cifra de control 5. Cum numerele cu aceeași cifră de control reprezintă o progresie aritmetică cu rația 9 , atunci raspunsul este (10000-100)/9=1100.
14. Răspuns corect: e) 148

Indicații: Grupăm elementele câte două (primul cu al doilea, al treilea cu al patrulea etc). Pentru prima pereche de elemente se face o comparare, presupunându-se că minimul perechii este minimul global, respectiv maximul perechii este maximul global. Pentru fiecare dintre celelalte perechi, se face o comparare ca să se determine cel mai mic și cel mai mare număr din pereche și apoi minimul perechii cu minimul global și maximul perechii cu maximul global. Deci, pentru fiecare pereche de elemente, se fac 3 comparații. Deci $3 \cdot \frac{100}{2}-2=148$
15. Răspuns corect: d) 4

Indicații: Dacă aranjăm nodurile astfel:
![](https://cdn.mathpix.com/cropped/2025_04_17_46e04c6acd873ea9558dg-335.jpg?height=286&width=356&top_left_y=1792&top_left_x=1139) componentă conexă conține nodurile de pe două coloane. Adică, prima componentă conține nodurile de pe coloanele 1 și 5 , a doua componentă conține nodurile de pe coloanele $\mathbf{2}$ și 6 , a treia de pe coloanele 3 și 7 iar a patra de pe coloanele 4 și 8 . Deci componentele conexe sunt:
\end{verbatim}

\{100, 96, 92, 88, 84, 80, ......, 12, 8, 4\}\\
\{99, 95, 91, 87, 83, 79, ......., 11, 7, 3\}\\
\{98, 94, 90, 86, 82, 78, ...., 10, 6, 2\}\\
\{97, 93, 89, 85, 81, 77, ....., 9, 5, 1\}

\begin{verbatim}

\end{verbatim}


\end{document}